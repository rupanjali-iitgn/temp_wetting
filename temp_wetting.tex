
\documentclass[a4paper,12pt,single,pdftex]{scrartcl}
\usepackage{ngerman}
 \usepackage{color}  
 \usepackage{html}  
 \usepackage{times}  
 \usepackage{graphicx} 
 \usepackage{fancyheadings}  
 \usepackage{hyperref}  
 \setlength{\parindent}{0.6pt} 
 \setlength{\parskip}{0.6pt} 
 \title{Designing solid surfaces with desired temperature dependence of wetting behavior}
 

\begin{document} 
\maketitle
\newpage

\label{ID_1723255651}\label{ID_268635931}\section{Scientific rationale}

\label{ID_1616607157}\subsection{Applications}

\label{ID_564676847}\subsubsection{Wetting engines}

\label{ID_1321830436}\subsubsection{Micro or nanofluidics}

\label{ID_278738019}\subsubsection{Miscellaneous}

\label{ID_1903445769}\paragraph{Catalysis}

\label{ID_731793233}\paragraph{Understanding Hysterisis}

\label{ID_235950698}\subsection{Current technical capabilities}

\label{ID_1442355972}\subsubsection{Experimental}

\label{ID_474314366}\subsubsection{Theoretical}

\label{ID_734735562}\subsubsection{Computational}

\label{ID_1110611260}\section{Background}

\label{ID_1392626561}\subsection{Research status (international)}

\label{ID_869146050}\subsubsection{Experimental}

\label{ID_121688724}\paragraph{Work with flat interfaces and chemical functionalization: Neumann et al.}

\label{ID_1314084600}\paragraph{Work with microcalorimetry}

\label{ID_30644458}\paragraph{Enthalpy-entropy compensation or "independence" of the two}

\label{ID_1521593594}\subsubsection{Computational}

\label{ID_1690050024}\paragraph{Work from Errington group}

\begin{itemize}
\label{ID_473109498}\item Systems that were studied
\label{ID_1487682070}\item How was it explained
\end{itemize}
\label{ID_473109498}\label{ID_1487682070}\label{ID_980454220}\paragraph{Work from Horsch group}

\label{ID_1854553780}\paragraph{Work of Haiping Fang group on the effect of lattice structure}

\label{ID_314119525}\paragraph{DFT calculations from Berim and Ruckenstein}

\label{ID_1942058277}\subsubsection{Theoretical}

\label{ID_896680657}\paragraph{Capillary waves can be understood as the undulations of the liquid-vapor interface in the direction perpendicular to the macroscopic interface. Let us say that our system is 3D with l-v interface in x-y plane, then the l-v interface for a particular confguration of molecules can be defined as z = \xi (x,y). Let z = 0 then represent the averaged l-v interface . The undulations of this surface is then analyzed by performing the spectral analysis of \xi as follows: Here, r is the vector  in the x-y plane. As k increases, the corresponding terms denote the contributions from the undulations of short wavelength. The capillary wave theory, defines the CWs as the undulations of very large wavelength, that is k \rightarrow 0. The macroscopic surface tension  \sigma can be expressed as the sum of the "bulk " surface tension  \sigma_b and the one coming from the CWs \sigma_c. [Refer Phys. Rev. A, 33, 1948 (1986)] It is expected that \sigma_c &lt; 0 because undulations are entropically favorable and they decrease the interfacial free energy.  It is also possible to express a k-dependent surface tension \sigma_k which excludes the contributions from undulations below a certain vector k.}

\label{ID_49666523}\paragraph{Kayser theoretically analyzed \sigma_k for temperatures near triple point of liquids and those near critical point. He observed that near the triple point, excluding the CWs of wavelength of about 10 times the molecular diameter can result in the \sigma_k which is about 15% higher than the macroscopic value. For temperatures near Tc, the similar increase of 15% is obtained if one excludes the CWs of wavelength larger that 20 times the molecular diameter. Excluding wavelengths greater than \lambda is same as restricting or pinning the l-v interface at distances separated by \lambda along xy plane. (See Adjacent figure). This effect has been confirmed by experimental studies [ Nature, 403, 871 (2000) ]}

\label{ID_1905060815}\paragraph{Importance of correlations along the interface}

\begin{itemize}
\label{ID_801342251}\item How it is affected by the temperature (via scaling laws)
\label{ID_1964284085}\item How they affect interfacial free energies
\label{ID_751345006}\item How they affect the temperature dependence
\end{itemize}
\label{ID_801342251}\label{ID_1964284085}\label{ID_751345006}\label{ID_1590536878}\paragraph{Generality of this phenomenon and its physics for all surfaces and fluids}

\label{ID_398274930}\paragraph{Factors that can influence transverse correlations}

\begin{itemize}
\label{ID_580082896}\item Which of these predominantly influence gamma_sl
\label{ID_874559696}\item Which predominantly influence dgamma_sl/dT
\end{itemize}
\label{ID_580082896}\label{ID_874559696}\label{ID_19652608}\subsubsection{Solvation and capillary waves}

\label{ID_106467715}\paragraph{Consider the change in the free energy of a system when a l-v interface of a solvent is brought in contact with a solute that is fixed in space. Let the entire system be in gran canonical ensemble. Therefore difference in free energy is given as follows:}

\label{ID_1903608681}\paragraph{Here the superscripts f and i denote the state with solute in contact with l-v interface and the one where it is in contact with saturated vapor, respectively. Let us denote all the similar differences in extensive quantities by operator Delta,. Following from the second law, the above equation can then be written in a differential form as follows:}

\label{ID_1377597507}\paragraph{We assume here that the solvent is at liquid-vapor coexistence conditions and that the system is at constant volume. Then the temperature derivative of free energy is given by}

\label{ID_863437212}\paragraph{Note that if the l-v interface was initially not influenced by any external field, then under saturation condition, the free energy change due to its rise upto the solute will be zero. In other words \Delta \mu above, only represents the change due to solute. Therefore \Delta S and \Delta N_s_o_l are mainly due to the fixed solute. The entropy change may be further divided into the contributions coming from the solute itself (i), solute-solvent interactions (uv) and solvent-solvent interactions (v v). The first part depends on the temperature and the intramolecular interactions of the solute. The second part consists of fluctuations in the solute-solvent interaction energy. The final part is due to the changes in the solvent-solvent interactions that are brought about by the solute. Note that the \Delta S_{uv} also contains the indirect effect of solvent-solvent changes on solute-solvent fluctuations. Typically, the "structural" changes in the solvent brought about by the solute are included in the third term, whereas the "effect of fluctuations in the solvent" is accounted in the second term. The density-density correlations along liquid-vapor interface are "fluctuations" and therefore they contrubte to the second term.}

\label{ID_1074911865}\paragraph{Since the free energy change associated with the rise of liquid-vapor interface is zero, we can look at the above process in terms of inserting a solute at the position r in the existing liquid-vapor interface. The related free energy change and its temperature derivative is given as follows:}

\label{ID_388366021}\paragraph{Here v_{\Psi} denotes the partial derivative of the solute-solvent interaction potential with respect to the coupling parameter \lambda. Starting from the above expressions, \delta S can be expressed as follows: Here, &lt;...&gt; denote the sensemble average taken over the entire system and &lt;..&gt;_1 denote the average taken over a system with solvent molecule fixed at r_1.  Note that \Phi and V are the contributions from the solvent-solvent interactions and solute-solvent interactions, respectively. M denotes the contributions from the chemical potential of the solvent . [TO CHECK: From our previous calculations with water and a model solvent, we observed that the for different distances from the solute \Phi and V are negligible within simulation uncertainties or very small.]}

\label{ID_349834534}\paragraph{we are assuming here that the solute is fixed and its internal degrees of freedom are not influenced by temperature. Also, we can express M in terms of density-density correlations. The different entropy components can then be expressed as follows: Where, \chi (r_1) is the local compressibility of the solvent at position r_1}

\label{ID_1103070320}\paragraph{In order to understand the second term, We note that \Delta N_{sol} is also related to the compressibility of the solvent at the liquid-vapor interface as follows (refer cap_wave_fields2):  This can be easily derived from the above equations. Since the function v_{\Psi} strongly decreases with the distance from the solute, and the solute is positioned at the liquid-vapor interface, all the integrals above are mostly influenced by solvent moleculecules near the solute.}

\label{ID_1305040262}\paragraph{ It is known that the major contributon to \chi comes from transverse correlations along liquid-vapor interface, so called capillary waves. Therefore CWs affect the temperature dependence predominantly via the terms containing \chi as follows: Further observe that the effect is related to how liquid-vapor saturation properties of the solvent change with temperature. [TO CHECK: For studies with water and a model solvent, we did observe that unlike other terms, this term is statistically significant. Thus, transverse correlations play an important role in the temperature dependence of the free energy change associated with a single solute fixed at the liquid-vapor interface.]}

\label{ID_276381896}\paragraph{ We now turn towards solid surfaces. In the above equations, the solute acts as a static field and therefore, they are even valid for more than one solute molecule fixed at different locations in space. As we gradually increase the density of the solute molecules, the "liquid-vapor interface" will start  resembling the solid surface more. The free energgy change \Delta \mu can then be expressed as follows: Here A is the interfacial area and \gamma_{ij} represents the interfacial free energies between phases i and j. Therefore, to understand the temperature dependence of \gamma_{sv} - \gamma_{sl} for anatomistically detailed solid surface, we should analyze the terms in the above expressions}

\label{ID_374576713}\subsection{Research status (national)}

\label{ID_135785427}\subsubsection{Chinmay's group compares sessile drop and calorimetric approaches}

\label{ID_1532965186}\paragraph{Useful for justifyng the microcalorimetry}

\label{ID_838488932}\paragraph{Can coating density affect the temperature dependence?}

\label{ID_1401479645}\paragraph{Can mix of different coatings be tested to "tune" hydrophilic(phobic) nature?}

\label{ID_1979164963}\subsection{Novelty of our work}

\label{ID_1997632539}\subsubsection{How realistic our models should be?}

\label{ID_625214648}\subsubsection{Exploiting transverse correlations}

\label{ID_1607535616}\subsubsection{New ways of studying wetting transitions for technologically relevant systems via experiments}

\label{ID_825518109}\paragraph{Emphasis on trends with a particular order parameter rather than measurements}

\label{ID_1028951925}\paragraph{Is it possible to create "a set of simple models" to consider a porous structure?}

\begin{itemize}
\label{ID_1602351942}\item New possible project
\label{ID_714004449}\item Alternatives to characterizing surface with Fowkes equation
\end{itemize}
\label{ID_1602351942}\label{ID_714004449}\label{ID_30899692}\subsubsection{Considering theories developed for flat surfaces in the context of more realistic and applied topic of porous powders}

\label{ID_1615838882}\section{Work plan}

\label{ID_1509493354}\subsection{Methods}

\label{ID_176697614}\subsubsection{Experimental}

\label{ID_222214319}\paragraph{Systems}

\begin{itemize}
\label{ID_890750199}\item "Packed-bed" and flat-surface wetting studies on crystalline materials
\end{itemize}
\label{ID_890750199}\label{ID_1528992532}\paragraph{Method of manipulating the hydrophilicity and hydrophocity}

\begin{itemize}
\label{ID_630039349}\item Radiation?
\label{ID_1162892636}\item Chemical treatment?
\label{ID_1113661769}\item Coating?
\end{itemize}
\label{ID_630039349}\label{ID_1162892636}\label{ID_1113661769}\label{ID_768785223}\paragraph{Contact angle determination on flat surface}

\label{ID_1121131843}\paragraph{Immersion calorimetry}

\label{ID_627706002}\paragraph{Surface area determination by some methods}

\label{ID_872435313}\subsubsection{computational}

\label{ID_1103117492}\paragraph{Systems}

\begin{itemize}
\label{ID_190203064}\item Models of different crustal faces like 100, 001 etc.
\label{ID_808283525}\item Solutes positioned at l-v interfaces
\end{itemize}
\label{ID_190203064}\label{ID_808283525}\label{ID_1619461814}\paragraph{Simulations of different types of crystal faces. (These will be validated against the experimental studies)}

\label{ID_1547495769}\paragraph{Simulations with solutes positioned at l-v interfaces}

\label{ID_1538917616}\paragraph{Theoretical analysis connecting the above two types of systems}

\label{ID_581300306}\subsection{Time schedule}

\label{ID_1195711759}\subsection{Use of research outcome}

\label{ID_1268337594}\subsection{Risk analysis}

\label{ID_392987275}\subsection{}

\label{ID_3125732}\section{Expertise and preliminary results}

\label{ID_1455861183}\subsection{Computational}

\label{ID_485443609}\subsubsection{For a ghost solute, how ensemble averages of Equation (25) of cap_wave_field1 paper change with temperature for bulk liquid and interface}

\label{ID_577294012}\subsubsection{How the contribution from capillary waves (deldelmu) affected by temperature?}

\label{ID_1474088810}\subsubsection{Studies with 2 solutes with different separation at bulk and interface: 1) How coeffs of Equation (25) ar affected by distance and 2) How they are affected by temperature}

\label{ID_1367637509}\subsubsection{Temperature dependence of wetting behavior on a lattice surface different than the one earlier observed}

\label{ID_1178677590}\subsection{Experimental}

\label{ID_1995511474}\subsubsection{measuring contact angles and temperature dependence on flat surfaces}

\label{ID_853592622}\subsubsection{Measurin wetting properties in porous beds}

\label{ID_1635488399}\subsubsection{Analysis of crystal faces and working with crystals using processes like milling}

\label{ID_1082352398}\section{Money and technical resources}


\newpage
%\tableofcontents

\end{document}
