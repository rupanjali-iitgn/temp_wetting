\section{1 Scientific rationale}

\par 1.1 The interaction between a solid surface and a liquid plays an important role in several natural and synthetic processes. An important phenomena in such systems is the wetting of a solid surface by liquid. The technological advances have enabled the development and application of solid-liquid interfaces  of increasingly microscopic dimensions. Further, there is a great scope for fine-tuning the chemical nature of solid surface at the molecular scale to achieve desired wetting behavior. Theoretical studies and computer simulations can help guide such manipulation. In this particular project we focus on the temperature dependence of the solid-liquid interfacial properties. Specifically, we will employ statistical mechanics to design solid surfaces that show desired thermal response of wetting. This will be achieved with the help of advanced molecular simulation techniques.
\par 1.2 Broadly speaking, the temperature dependence of solid-liquid interfacial properties are important where the solid-liquid interface plays an important role. In microfluidics and nanofluidics wetting behavior is very important and there are numerous scenarios where such devices encounter environments with varying temperatures. For example, microfluidic devices used in processes involving endothermic or exothermic reactions. Another important application is the role of temperature in the wetting of ink droplets on solid surfaces. With the application of inkjet printing technology in areas like microelectronics and solarcells it becomes important to understand how the wetting behavior of ink droplets depends on temperature \textbackslash cite\{soltman2008inkjet\}. The knowledge about dynamics of heat and mass transfer is indeed crucial in the development of such realistic systems. However, it is useful to first study such systems under equilibrium conditions to understand the factors that must be considered for designing them. Moreover, such a study can provide useul insights into the underlying molecular-scale phenomena. We explain this point with the example of thermal engines based on the wetting behavior of solids.\textbackslash cite\{AIC:AIC690490320\} These engines are based on a thermodynamic cycle made of processes that involve the spreading of liquid on a solid surface and have applications in micro-mechanical actuators \textbackslash cite\{xu2011conceptual\} and car dampers \textbackslash cite\{eroshenko2007new\}. .
\par 1.3 If one assumes that all the involved processes are reversible then it is straightforward to relate the technologically relevant quantities of the engine to molecular-level details of the system. For example, the exchange of work and heat with the surroundings is related to the interfacial free energies (solid-liquid, liquid-vapor and solid-vapor) and their derivatives with respect to temperature, respectively. Using the framework of statistical mechanics these quantities can be further related to the chemical nature of solid surface and liquid. Thus, by designing solid surfaces that result in the desired thermal response of the wetting behavior one can, in principle, develop wetting engines having desired efficiency
\par 1.4 With regards to the nature of solid surface, there are several factors that can potentially affect solid-fluid interfacial free energiesand their temperature derivatives. In this project we are partiularly interested in the effect of heterogenity along the plane of the solid surface. Such a heterogenity can be due to the variation of the chemical nature or due to the surface roughness. Moreover,  it can range over multiple scales. For example, roughness can be considered at the atomistic level due to a particular arrangement of surface atoms, whereas at larger scales it can be due to variation in the macroscopic topography of the surface.
\par 1.5 Such heterogeneity affects the solid-fluid interfacial properties by influencing the correlations between local densities of the fluid molecules along the plane parallel to the surface. Therefore, in order to rationally design such a heterogeneous surface it is important to understand the contribution of abovementioned correlations to solid-fluid interfacial free energies and their derivatives with respect to temperature. In this document we refer the density-density correlations in liquid along liquid-vapor or solid-liquid interfaces as transevrse correlations (abbreviated as TC)\section{2 Research status (international)}
\subsection{2.1 Theoretical}

\par 2.1.1 The temperature dependence of wetting phenomenon has been closely connected with the study of wetting  from the perspective of phase coexistence. Let us consider two states of a system: 1) a solid-vapor interface being partially wet by the liquid and 2) solid-vapor interface being completely wet. It is possible to study these states as two surface "phases". Similar to coexistence between liquid and vapor these two surface phases can be in coexistence at certain system parametrers. The parameters of interest are temperature, or those related to the nature of the solid surface like the strength of interaction between solid and liquid, surface roughness etc. Analogous to wetting transitions, one may consider drying transitions where the role of liquid and vapor in the above two states are interchanged. The Figure \textbackslash ref\{figures/wet\_dry\_trans.eps\} denotes the wetting and drying transitions.
\par 2.1.2 In1977 Cahn proposed that there exists a temperature below the critical temperature of fluid at which the wetting transition on a given solid surface occurs \textbackslash cite\{cahn1977critical\}. The argument was based on the comparison between the temperature dependence of \$\textbackslash gamma\_\{\textbackslash mathrm\{sv\}\} - \textbackslash gamma\_\{\textbackslash mathrm\{sl\}\}\$  and \$\textbackslash gamma\_\{\textbackslash mathrm\{lv\}\}\$. Cahn used a theoretical approach called as square-gradient version of Van der Waals theory and did not consider the atrraction between solid and liquid in explicit detail.
\par 2.1.3 Later, the square-gradient theory was modified by Nakanishi and Fisher to include the dependence of solid-fluid attraction \textbackslash cite\{nakanishi1982multicriticality\}. They proposed a phase diagram which represented the coexistence lines for the wetting and drying transitions in the plot of temperature vs a parameter denoting the strength of solid-fluid interaction. Their phase diagram shows that for weak solid-liquid interaction, increase in temperature brings the system closer to drying transition. Whereas for strong solid-fluid interactions, increase in temperature brings the system closer to the wetting transition. It should be noted, however, that  these theories do not consider the molecular-level details of the solid surface or liquid. Therefore, the observations do not provide quantitative predictions of the wetting behavior in real systems. Secondly, the above studies only understand the behavior of systems "at" the wetting and drying transitions. From a technological perspective, it is also important to understand the temperature dependence of contact angles in the partial wetting regime.
\par 2.1.4 Another theoretical approach that is employed to understand the wetting behavior is claasical Density Functional theory (DFT). This theory considers molecular-level details of the system but is computationally less expensive as compared to molecular simulations. It considers the system in grand canonical ensemble and expresses the grand potential as a functional of the space-dependent density of the fluid. The equilibrium density of the fluid is calculated by minimizing the grand potential. The interfacial free energies like \$\textbackslash gamma\_\{\textbackslash mathrm\{sv\}\}\$,  \$\textbackslash gamma\_\{\textbackslash mathrm\{sl\}\}\$ and \$\textbackslash gamma\_\{\textbackslash mathrm\{lv\}\}\$ are then obtained from the equilibrium density profile via the grand potential. Though this approach can be only used for very simple systems, it is able to provide qualitative understanding of the observations in real systems.
\par 2.1.5 One of the first applications of DFT to study wetting behavior is by Sullivan who employed the Van der Vaals theory without the square-gradient approximation \textbackslash cite\{sullivan1981surface\}. Particularly, the Cahn's definition of solid surface was reinterpreted in terms of a more realistic solid-fluid interaction potential. This enabled Sullivan to study the variation of contact angle \$\textbackslash theta\$ over a range of solid-fluid interaction strengths \$\textbackslash epsilo\_\{\textbackslash mathrm\{sw\}\}\$ at multiple temperatures. It was observed that for small \$\textbackslash epsilon\_\{\textbackslash mathrm\{sw\}\}\$ - where \$\textbackslash theta\$ is large - \$\textbackslash theta\$ increased with temperature. On the other hand, for large \$\textbackslash epsilon\_\{\textbackslash mathrm\{sw\}\}\$ \$\textbackslash theta\$ decreased with temperature. Similar influence of solid-fluid interactions on the temperature dependence of \$\textbackslash theta\$ was also observed in more recent DFT studies. Kuiper and Blokhuis used DFT to reinterpret the square-gradient model of Nakanishi and Fisher described earlier in terms of the realistic solid-fluid interactions \textbackslash cite\{kuipers2009wetting\}. In addition to verifying the phase diagram of wetting and drying transitions, they computed the contact angles for a wide range of \$\textbackslash epsilon\_\{\textbackslash mathrm\{sw\}\}\$. Here, \$\textbackslash epsilon\_\{\textbackslash mathrm\{sw\}\}\$ was characterized by the depth of square-well potential between solid and fluid. As in study of Sullivan, contact angles increased with temperature for small  \$\textbackslash epsilon\_\{\textbackslash mathrm\{sw\}\}\$ whereas opposite trend was observed for large  \$\textbackslash epsilon\_\{\textbackslash mathrm\{sw\}\}\$. Recently, Berim and Ruckenstein employed the canonical version of DFT to study the temperature dependence of contact angles over a range of substrate-fluid interaction strengths \textbackslash cite\{berim2011universality\}. The system they studied was a liquid drop on a solid surface instead of a liquid film in contact with surface as studied by earlier mentioned works. The contact angle was also measured visually at the three phase contact line. Their solid surface was more realistic than the earlier studies and consisted of discrete molecules interacting with fluid molecules via Lennard Jones (LJ) potential. The LJ parameter for the solid-fluid interaction characterized \$\textbackslash epsilon\_\{\textbackslash mathrm\{sw\}\}\$. Like earlier studies, they observed that \$\textbackslash theta\$ increased with temperature for low \$\textbackslash epsilon\_\{\textbackslash mathrm\{sw\}\}\$ whereas opposite trends were observed for large \$\textbackslash epsilon\_\{\textbackslash mathrm\{sw\}\}\$. The notable distinction from the earlier studies was that the transition from the former regime to the later regime happened over a very narrow range of \$\textbackslash epsilon\_\{\textbackslash mathrm\{sw\}\}\$. Berim and Ruckenstein also commented on the significance and the potential applications of the intermediate substrate-strength at which the \$\textbackslash theta\$ is almost independent of temperature. The above studies indicate that the temperature dependence of \$\textbackslash theta\$ show some general trends with respect to the strength of attraction between surface and fluid. That being said, the complexity of systems that can be handled using DFT is limited. Especially, the heterogeneity of the solid surface that may play an important role in the wetting behavior was not considered.
\par 2.1.6 We finally note that the observed phase diagrams of wetting-drying transitions and the influence of \$\textbackslash epsilon\_\{\textbackslash mathrm\{sw\}\}\$ are related. The latter behavior may be roughly predicted from the phase coexistence curves in the T vs \$\textbackslash epsilon\_\{\textbackslash mathrm\{sw\}\}\$ plane as shown in Figure \textbackslash ref\{figures/wet\_dry\_phasediagram.eps\}. One may observe that as T increases, the range of \$\textbackslash epsilon\_\{\textbackslash mathrm\{sw\}\}\$ corresponding to the partial wetting regime narrows down and therefore the plots of \$cos\textbackslash theta\$ vs \$\textbackslash epsilon\_\{\textbackslash mathrm\{sw\}\}\$ becomes steaper as recently commented by Henderson \textbackslash cite\{henderson2011discussion\}. However, it is not trivial to explain the very narrow range of \$\textbackslash epsilon\_\{\textbackslash mathrm\{sw\}\}\$ over which the inversion in T-dependence occurs. Another important point is that apart from the effect of \$\textbackslash epsilon\_\{\textbackslash mathrm\{sw\}\}\$, very few theoretical studies have attempted to understand the temperature dependence in terms of the role played by the molecular scale phenomena. Most of the arguments invoke the macroscopic physics of the system, say related to the surface phase coexistence. Though it is just a matter of interpretation, explaining the observations in terms of molecular scle phenomena can aid in the technological application of various observations.\subsection{2.2 Experimental}

\par 2.2.1 Wetting of liquids on solid surfaces is a vast field with numerous applications. In recent years, many experimental studies have documented wetting properties on natural and artificial surfaces. This trend is fuelled by the improved ability to manipulate microscopic and nanoscopic details of the surfaces and the measuring devices. One of the most important sub-field of the wetting research is the investigation of solid surfaces that have superhydrophobic properties. In most studies, the equilibrium property that is measured is the contact angle (\$\textbackslash theta\$) that a small liquid drop makes with the solid surface. In others, instead of liquid drop the force of adhesion between liquid and a solid surface is measured, or the time required for absorption of a liquid in a column packed with the desired solid material is determined. The later two methods then use theoretical models to relate the measured quantities to \$\textbackslash theta\$. Most of these studies use \$\textbackslash theta\$ to characterize the hydrophobicity or hydrophilicity of the solid surfaces. Generally, the main goal of such studies is to design a solid surface by manipulating a single feature of solid surface, either roughness \textbackslash cite\{ramiasa2016tuning\} or solid-fluid interaction \textbackslash cite\{wang1999studies\}. Relatively fewer studies have been performed to understand the wetting behavior where combinations of two or more surface features are considered \textbackslash cite\{liu2014bio\}. Coming to the temperature dependence, the available experimental data is even rare. The most prominent set of data is by Neumann and colleagues \textbackslash cite\{neumann1970temperature,neumann1974advances,budziak1991temperature\}, who studied the temperature dependence of contact angles on different polymer surfaces. It was observed that the contact angles may increase or decrease with the nature of the solid surface, especially the strength of interaction between solid and fluid. These results approximately comply with the theoritical predictions \textbackslash cite\{sullivan1981surface,nakanishi1982multicriticality,berim2011universality\}. However, it should be noted that the systems used in these experiments are more complex than those generally used in theoretical investigations. The complex nature of the solid surface meant that many factors related to the surface may have contributed to the observed temperature dependence. For example, in addition to changes on the liquid side of the solid-liquid interface, polymer surfaces may have been significantly influenced by change in temperature.  Recently, there have been studies concerned with the development of solid surfaces which can switch from hydrophilic to hydrophobic behavior with increase in temperature \textbackslash cite\{balamurugan2003thermal,de2014reversible,li2014thermo\}. These surfaces are made by grafting certain polymers on a solid base. The grafted polymers undergo transition from the extended state to the collapsed state on increasing temperature, which changes their interaction with the liquid molecules, therby resulting in the change of wetting behavior. Thus, the observed thermal response of wetting is actually the manifestation of the phase behavior of polymers in a particular liquid. In the proposed project we plan to first decopule the thermally induced changes occuring in the solid surface from those in the liquid. We will only study solid surfaces whose structure is negligibly influenced by the changes in temperature or by the liquid. Therefore, we do not consider solid surfaces grafted with thermal responsive polymers.
\par 2.2.2 Apart from measurement of contact angles, the wetting behavior on solid surfaces have also been studied by looking at the growth of a thin liquid film on the surface of interest under controlled conditions. This approach is more in line with the theoretical studies concerned with wetting and drying transistions \textbackslash cite\{bonn2001wetting\}. Typically, the eqilibrium thickness of a liquid film on a surface indicates whether the system is in complete drying (\$\textbackslash theta = \textbackslash pi\$), partial wetting or complete wetting (\$\textbackslash theta = 0\$) regime. The transitions between each of these regimes can be considered as surface phase transitions akin to those between bulk phases, say liquid and vapor. It is then possible to represent these regimes in a surface phase diagrams in the plane of temperature vs some order parameter characterizing the surface feature of interest. This order parameter may be related to attraction between surface and fluid molecules \$\textbackslash epsilon\_\{\textbackslash mathrm\{sw\}\}\$ or roughness   Several such studies have been performed with goal of studying the fundamental aspects of wetting transitions \textbackslash cite\{bonn2001wetting\}. Most of these studies consider the temperature temperature dependence because it is the system variable that is used to study the transition. However, due to the technical challenges involved in performing such studies, the solid surfaces they investigate are relatively less complex than those employed in the experiments measuring contact angles. As a result, to our knowledge, no systematic data is available as a function of either \$\textbackslash epsilon\_\{\textbackslash mathrm\{sw\}\}\$ or roughness. Unlike wetting transitions, drying transitions have not been studied using experiments. Such studies can be useful in understanding the thermal response of wetting on solid surfaces where contact angles are large. That being said, in the proposed project we do not plan to experimentally study the wetting and drying transitions using the above approach. This is because at this stage, there are several technical challenges in employing these methods to study the solid-fluid systems that we wish to consider. Moreover, it is impractical to employ such techniques to study wetting on many different types of solid surfaces. Finally, we note that unlike solid surfaces, the technical challenges in using the "growth of liquid film" approach  to study wetting on liquid-vapor or liquid-liquid interfaces are less severe. However, such systems are outside the scope of the current project.\subsection{2.3 Computational}

\par 2.3.1 Here we focus our discussion on the Molecular Dynamics (MD) or Monte Carlo (MC) simulation studies performed to study equilibrium wetting behavior of liquids on solid surfaces. The number of such studies have grown in previous few years to predict the wetting behavior on novel surfaces that can be manipulated at increasingly smaller scales. The trend has also been fuelled by the discovery of new applications for the novel substances like room temperature ionic liquids (RTILs) and graphene. The molecular simulations provide two main advantages: 1) They can be used to estimate the wetting behavior in systems that are still to be developed and therefore, guide the rational design of surfaces. This is important because as the scale of features on a surface decreases, the conventional theories that are mostly based on continuum approximation fail.  2) They can be used to understand the molecular-level aspects of the wetting behavior that is observed in experiments. The important challenges in any molecular simulation study are as follows: 1) Upper limit on the system size due to steep increase in the required computational resources. Therefore, as will be done in the proposed study, they are only limited to the systems with sizes less than few nanometers. 2) The results depend on the models of molecules. Ideally, it is expected that the models are parameterized to reproduce the experimental properties. However, if prediction of properties is the objective, such parameterization is not possible. In the current study we do not focus on the the quantitative prediction of properties for real systems. Also, we do not aim to rigorously understand the molecular level phenomena of real systems that are used in experiments. Our objective is to gather the general trends in the macroscopic properties by varying certain features of the solid surface. This allows us to use simple models for molecules of liquids and solids and work at conditions that enable effeicient calculation of desired properties.
\par 2.3.2 There are two important ways of computing the wetting properties like interfacial free energies and contact angles from molecular simulations. One approach is by actually simulating a nanodroplet of a liquid on a solid surface. This strategy is generally employed via MD simulations. This is analogous to the sessile drop method in experiments.  The contact angle is then determined from the angle that the liquid-vapor interface makes with the solid surface at the three phase contact line. One important difference from the experiments is that the interfaces can be more precisely defined based on the local fluid densities. The solid-liquid (\$\textbackslash gamma\_\{\textbackslash mathrm\{sl\}\}\$)  and solid-vapor (\$\textbackslash gamma\_\{\textbackslash mathrm\{sv\}\}\$) interfacial free energies are then calculated using the Young's equation. The main limitations of this approach are: 1) Large systems are necessary to account for the effect of line tensions \textbackslash cite\{indekeu1993line\} on the observed values of the contact angle, and 2) Discrepancies may be observed at high temperatures if the droplet is not in equilibrium with the surrounding vapor. This method was recently employed to determine the contact angles of a monoatomic liquid on solid surfaces at different temperatures. \textbackslash cite\{becker2014contact,horsch2010contact\}. In these studies, the authors studied surfaces having different parameters for the solid-fluid interaction strengths \$\textbackslash epsilon\_\{\textbackslash mathrm\{sw\}\}\$. They investigated the temperature dependence of all the three wetting regimes: complete drying, partial wetting and complete wetting. They observed that the contact angle \$\textbackslash theta\$  increase with temperature for small \$\textbackslash epsilon\_\{\textbackslash mathrm\{sw\}\}\$ and decreases with temperature for large \$\textbackslash epsilon\_\{\textbackslash mathrm\{sw\}\}\$. Importantly, they observed that the inversion of temperature dependence happens sharply at \$\textbackslash epsilon\_\{\textbackslash mathrm\{sw\}\}\^o\$ where \$\textbackslash theta = \textbackslash mathrm\{90\^o\}\$ and the behavior is symmetric about \$\textbackslash epsilon\_\{\textbackslash mathrm\{sw\}\}\^o\$. This observation was explained by referiing to the linear dependence of solid-fluid interfacial free energies on fluid densities and \$\textbackslash epsilon\_\{\textbackslash mathrm\{sw\}\}\$. The opposite trends with temperature for weak and strong surfaces are expected from theory, as both wetting and drying transitions are favored at high temperatures. However, from a statistical-mechanical perspective, one neither expects the inversion to occur at a single value of \$\textbackslash epsilon\_\{\textbackslash mathrm\{sw\}\}\^o\$ nor that it should correspond to the \$\textbackslash theta = \textbackslash mathrm\{90\^o\}\$.
\par 2.3.3 The second way of applying molecular simulations is by using an open system (grand canonical ensemble) where the number of fluid molecules is variable. Such a simulated system is compliant with the statistical-mechanics-based theories on wetting because at microscopic scales, the interface exchanges fluid molecules with the surroundings. Moreover, it helps in maintaining liquid-vapor coexistence conditions that are necessary to study equlibrium wetting phenomenon. Instead of simulating a nanodroplet, one tracks the variation of free energy as a liquid film is grown on a solid surface after starting from the saturated vapor phase \textbackslash cite\{grzelak2008computation\}. The properties like \$\textbackslash gamma\_\{\textbackslash mathrm\{lv\}\}\$ and \$\textbackslash gamma\_\{\textbackslash mathrm\{sv\}\} - \textbackslash gamma\_\{\textbackslash mathrm\{sl\}\}\$ can then be computed from the above profile. Contact angles are then computed by using the Young's equation. This method is generally applied via MC simulations and is compatible with various "Alchemical" techniques \textbackslash cite\{kumar2011monte\}. These methods have been previously employed to study the wetting behavior in different solid-fluid systems at different temperatures \textbackslash cite\{rane2011monte,kumar2011monte,rane2014understanding\}. The calculations were performed over a wide range of solid-fluid interaction strengths \$\textbackslash epsilon\_\{\textbackslash mathrm\{sw\}\}\$. Similar to the studies described in the previous paragraph, it was observed that for small \$\textbackslash epsilon\_\{\textbackslash mathrm\{sw\}\}\$ contact angles increase with temperature whereas opposite trend exists for large \$\textbackslash epsilon\_\{\textbackslash mathrm\{sw\}\}\$. However, the inversion of temperature dependence was not sharp, but occurred over a narrow range of \$\textbackslash epsilon\_\{\textbackslash mathrm\{sw\}\}\$. This range also depended on the chemical nature of the fluid and the solid surface, and in most cases, corresponded to \$\textbackslash theta\$ diiferent from \$90\^o\$. These obervations are in agreement with the theoretical predictions based on the density functional theory \textbackslash cite\{sullivan1981surface\}.
\par 2.3.4 The above mentioned molecular simulation studies indicate that there is good potential for manipulating the thermal response of the wetting behavior by modifying the nature of solid surface.\subsection{2.4 Role of TCs}

\par 2.4.1 The role of TCs in the temperature dependence of the wetting behavior can be understood in two ways. First, by looking at the contribution of CWs to the interfacial free energies at liquid-vapor-like interfaces. CWs are the undulations of the liquid-vapor interface in the direction perpendicular to the macroscopic interface which result in long-ranged TCs along the l-v interface. Let us say that our system is 3D with l-v interface in x-y plane, then the l-v interface for a particular confguration of molecules can be defined as z = \textbackslash xi (x,y). Let z = 0 then represent the averaged l-v interface . The undulations of this surface is then analyzed by performing the spectral analysis of \textbackslash xi as follows: Here, r is the vector  in the x-y plane. As k increases, the corresponding terms denote the contributions from the undulations of short wavelength. The capillary wave theory, defines the CWs as the undulations of very large wavelength, that is k \textbackslash rightarrow 0. The macroscopic surface tension  \textbackslash sigma can be expressed as the sum of the "bulk " surface tension  \textbackslash sigma\_b and the one coming from the CWs \textbackslash sigma\_c. [Refer Phys. Rev. A, 33, 1948 (1986)] It is expected that \textbackslash sigma\_c < 0 because undulations are entropically favorable and they decrease the interfacial free energy.  It is also possible to express a \textbackslash lambda-dependent surface tension \textbackslash sigma\_\{\textbackslash lambda\} which excludes the contributions from undulations above a particular wavelenght \textbackslash lambda
\par 2.4.2 Kayser theoretically analyzed \textbackslash sigma\_\{\textbackslash lambda\} for temperatures near triple point of liquids and those near critical point. He observed that near the triple point, when \textbackslash lambda is about 10 times the molecular diameter, \textbackslash sigma\_\{\textbackslash lambda\} is about 15\% higher than the macroscopic value. For temperatures near Tc, the similar increase of 15\% is obtained if \textbackslash lambda is 20 times the molecular diameter. The contributions from undulations of larger wavelength were also found to be important in different experimental studies [ Nature, 403, 871 (2000) ]. Here, \textbackslash sigma\_\{\textbackslash lambda\} was computed from the scattering intensity obtained from a carefully performed X-ray diffraction study of the l-v interface. A more thorough analysis of the temperature dependence of \textbackslash sigma\_\{\textbackslash lambda\} was recently performed by Hoefling and Dietrich using molecular simulations. [EPL, 109, 46002]. They observed that the ratio \textbackslash sigma\_\{\textbackslash lambda\}/\textbackslash sigma is significantly altered with temperature. There observations have also been supported by the theoretical study performed by Parry, Rascon and Evans.
\par 2.4.3  Excluding wavelengths greater than \textbackslash lambda is same as restricting or pinning the l-v interface at distances separated by \textbackslash lambda along xy plane. (See Adjacent figure). The conclusions from the above studies then imply that surface tension of such a pinned interface shows a different temperature dependence than that of the unpinned or free interface. In real systems, such a pinned interface can be realized in several scenarios like liquid-vapor interface confined inside a nanopore or the interface pinned between the colloidal particles. With regards to wetting, such a scenario also exists on certain solid surfaces. For example, several designs of superhydrophobic or superoleophobic surfaces have protruding stuctures having dimensions between micrometers and nanometers. When liquid sits at the top of these pillars (Cassie regime) a pinned liquid-vapor interface exists between the pillars (See the adjacent figure). Note that the macroscopic solid-liquid interface now consists of the combination of several such pinned liquid-vapor interfaces. The theoretical predictions about the temperature dependence of \textbackslash sigma\_\{\textbackslash lambda\} can therefore guide one to tune the distance between these pillars and their arrangement in two dimensions to obtain the desired thermal response of the wetting behavior.
\par 2.4.4 For more general solid surfaces which do not have such structures, the above analysis cannot be used because there are no liquid-vapor-like interfaces near the solid surface. Also for the solid surfaces with pillars, it is only valid when the distance between the pillars is much larger than the molecular diameter of the liquid. Additionaly note that the analysis will be very complicated for the solid surfaces with flexible structures. If the attraction between molecules of solid surface and those of liquid is weak then long-ranged TCs do exist near solid surface. This was recently shown for water molecules near hydrophobic surfaces using molecular simulations [PRL, 115, 016103 (2015)]. In this work Wilding and Evans used quantity called local compressibility \textbackslash chi as a function of the distance z from the solid surface. The large magnitude of \textbackslash chi (z) indicates the presence of long-ranged TCs. They observed a distinct peak near surfaces, even for those with contact angles as less as 26\^\{o\}. These results show that TCs can contribute to the solid-liquid interfacial properties. In the paragraphs below we explain how TCs may affect the temperature dependence of  s-l interfacial free energies.
\par 2.4.5 Write about studies that looked at the role of TDCs: 1) Velasco and Tarazona, J Chem Phys, 91,7916 (1989) 2) Oleinikova and Brovchenko, Phys. Rev. E, 76, 041603 (2007) and 3) Pleimling, Journal of Physics A, Mathematics and General, 37, R79 (2004)
\par 2.4.6 TDCs have not been experimentally studied. However, experiments have detected the depletion of liquid density near weak solid surfaces. From simulation and theoretical studies, such depletion generally accompanies the increased local compressibility.  However, the actual range of TDCs via experiments remains to be understood. Drying transition not been detected using Experiments. However, the depletion of liquid density near the solid surfaces of very weak strengths has been observed. Refer 1) Hess et al., Phys. Rev. Lett., 78, 1739 (1997) where the depletion of Ne liquid density on Cs surface was onbserved 2)Jensen et al. Phys. Rev. Lett., 90, 086101 (2003) also observed similar depletion of water near hydrophobic surface with Xray reflectiveity studies. They also correlated their observations with contact angle measurements 3) Langmuir, 23, 598 (2007) used neutron diffraction to observe similar depletion between water and hydrophobic solid surface. 4) Poynar et al. Phys. Rev. Lett.,97, 266101 (2006) using x-ray reflectivity measurements
\par 2.4.7 The depletion of liquid density near the solid surfaces of very weak strengths has been observed in experiments. Refer 1) Hess et al., Phys. Rev. Lett., 78, 1739 (1997) where the depletion of Ne liquid density on Cs surface was onbserved 2)Jensen et al. Phys. Rev. Lett., 90, 086101 (2003) also observed similar depletion of water near hydrophobic surface with Xray reflectiveity studies. They also correlated their observations with contact angle measurements 3) Langmuir, 23, 598 (2007) used neutron diffraction to observe similar depletion between water and hydrophobic solid surface.\subsection{2.5 Solvation and capillary waves}

\par 2.5.1 Consider the change in the free energy of a system when a l-v interface of a solvent is brought in contact with a solute that is fixed in space. Let the entire system be in gran canonical ensemble. Therefore difference in free energy is given as follows:
\par 2.5.2 Here the superscripts f and i denote the state with solute in contact with l-v interface and the one where it is in contact with saturated vapor, respectively. Let us denote all the similar differences in extensive quantities by operator Delta,. Following from the second law, the above equation can then be written in a differential form as follows:
\par 2.5.3 We assume here that the solvent is at liquid-vapor coexistence conditions and that the system is at constant volume. Then the temperature derivative of free energy is given by
\par 2.5.4 Note that if the l-v interface was initially not influenced by any external field, then under saturation condition, the free energy change due to its rise upto the solute will be zero. In other words \textbackslash Delta \textbackslash mu above, only represents the change due to solute. Therefore \textbackslash Delta S and \textbackslash Delta N\_s\_o\_l are mainly due to the fixed solute. The entropy change may be further divided into the contributions coming from the solute itself (i), solute-solvent interactions (uv) and solvent-solvent interactions (v v). The first part depends on the temperature and the intramolecular interactions of the solute. The second part consists of fluctuations in the solute-solvent interaction energy. The final part is due to the changes in the solvent-solvent interactions that are brought about by the solute. Note that the \textbackslash Delta S\_\{uv\} also contains the indirect effect of solvent-solvent changes on solute-solvent fluctuations. Typically, the "structural" changes in the solvent brought about by the solute are included in the third term, whereas the "effect of fluctuations in the solvent" is accounted in the second term. The density-density correlations along liquid-vapor interface are "fluctuations" and therefore they contrubte to the second term.
\par 2.5.5 Since the free energy change associated with the rise of liquid-vapor interface is zero, we can look at the above process in terms of inserting a solute at the position r in the existing liquid-vapor interface. The related free energy change and its temperature derivative is given as follows:
\par 2.5.6 Here v\_\{\textbackslash Psi\} denotes the partial derivative of the solute-solvent interaction potential with respect to the coupling parameter \textbackslash lambda. Note that \textbackslash rho\_v (r\_1 , r) denotes the density of the vapor around the solute molecule at the initial state corresponding to \textbackslash Omega\_i (r). Starting from the above expressions, \textbackslash delta S can be expressed as follows: Here, <...> denote the sensemble average taken over the entire system and <..>\_1 denote the average taken over a system with solvent molecule fixed at r\_1.  Note that \textbackslash Phi and V are the contributions from the solvent-solvent interactions and solute-solvent interactions, respectively. M denotes the contributions from the chemical potential of the solvent . [TO CHECK: From our previous calculations with water and a model solvent, we observed that the for different distances from the solute \textbackslash Phi and V are negligible within simulation uncertainties or very small.]
\par 2.5.7 The superscripte v denote the same quantities for the initial state. We are assuming here that the solute is fixed and its internal degrees of freedom are not influenced by temperature. Also, we can express M in terms of density-density correlations. The different entropy components can then be expressed as follows: Where, \textbackslash chi (r\_1) is the local compressibility of the solvent at position r\_1. Again, the superscripts  refer to the quantity corresponding to the initial state.
\par 2.5.8 In order to understand the second term, We note that \textbackslash Delta N\_\{sol\} is also related to the compressibility of the solvent at the liquid-vapor interface as follows (refer cap\_wave\_fields2):  This can be easily derived from the above equations. Since the function v\_\{\textbackslash Psi\} strongly decreases with the distance from the solute, and the solute is positioned at the liquid-vapor interface, all the integrals above are mostly influenced by solvent moleculecules near the solute.
\par 2.5.9  It is known that the major contributon to \textbackslash chi comes from transverse correlations along liquid-vapor interface, so called capillary waves. Therefore CWs affect the temperature dependence predominantly via the terms containing \textbackslash chi as follows: Further observe that the effect is related to how liquid-vapor saturation properties of the solvent change with temperature. [TO CHECK: For studies with water and a model solvent, we did observe that unlike other terms, this term is statistically significant. Thus, transverse correlations play an important role in the temperature dependence of the free energy change associated with a single solute fixed at the liquid-vapor interface.]
\par 2.5.10  We now turn towards solid surfaces. In the above equations, the solute acts as a static field and therefore, they are even valid for more than one solute molecule fixed at different locations in space. As we gradually increase the density of the solute molecules, the "liquid-vapor interface" will start  resembling the solid surface more. The free energgy change \textbackslash Delta \textbackslash mu can then be expressed as follows: Here A is the interfacial area and \textbackslash gamma\_\{ij\} represents the interfacial free energies between phases i and j. Therefore, to understand the temperature dependence of \textbackslash gamma\_\{sv\} - \textbackslash gamma\_\{sl\} for anatomistically detailed solid surface, we should analyze the terms in the above expressions\section{3 Research status (national)}

\par 3.1 The group headed by Jayant Singh at IIT Kanpur used molecular simulations to study equilibrium and dynamic aspects of the wetting behavior. Over the years, they have studied the influence of surface roughness \textbackslash cite\{metya2014wetting\} and the chemical nature of surface \textbackslash cite\{dutta2011wetting\} on the equilibrium wetting behavior. Most studies have employed the nanodroplet approach mentioned earlier and MD simulations to calculate contact angles for realistic models of liquids and surfaces. Some of their studies have also looked at the fundamental aspects of wetting transitions by tracking the growth of a liquid film via MC simulations \textbackslash cite\{khan2011surface\}. Recently, they and the collaborators looked at the thermal response of wetting behavior of water on a solid surface grafted with thermo-responsive polymer \textbackslash cite\{bhandary2016molecular\}. As noted earlier, such phenomena are due to the combined effects of rearrangement of surface molecules and the change in surface-fluid interactions. The work proposed in this document will be complementary to those studies because we will study only the effect of solid-fluid interaction usng surfaces whose molecules do not rearrange with temperature. Regarding the temperature dependence, Jayant Singh and collaborators have also studied the relation between wetting of water on nano-patterned surfaces and the nucleation of ice \textbackslash cite\{singh2014characterization\}. Though the proposed study will be concerned with fluids away from their triple point, we envision that nucleation can influence design decisions in the potential technologies involving wetting at different temperatures.
\par 3.2 We expect that the insights obtained from the proposed project will aid in rational design of solid surfaces with desired wetting behavior. The work of Indian researchers involved in fabricating such surfaces will be crucial for the real-world application of our ideas. Here, I will briefly describe the developments of some Indian research groups that are very relevant to the proposed project. At CSIR National Aerospace Laboratories, Harish Barshilia and co-workers are involved in fabrication and characterization of nano-patterned surfaces for different applications. Though most of these surfaces are developed with the the goal of improving the photocatalysis \textbackslash cite\{barshilia2012nanometric\}, they have also reported the wetting properties of some of them \textbackslash cite\{barshilia2014superhydrophobic,selvakumar2013vapor\}. An important part of their work is to study the thermal and mechanical stability of the resulting surfaces or coatings \textbackslash cite\{Chaliyawala201695\}. This is particularly important for improving the applicability of such surfaces in commercial technologies. Also, such "rigid" surfaces are similar to to ones that we plan to use in the proposed project.
\par 3.3 The Thematic Unit of Excellence for Nanofabrication at IIT Kanpur is involved in bottom-up as well as top-down approaches for fabricating solid surfaces with nanoscopic features. Among the participating groups, the one headed by Ashutosh Sharma is involved in creating nanostructures by exploiting the dewetting properties of thin ploymer films \textbackslash cite\{doi:10.1021/acs.langmuir.5b02977\}. The process also involves the variation of the temperature of the system. We think that insights into the thermal response of wetting behavior can help in achieving a finer control over the nanoscale features produced by such processes. Another group headed by Krishnacharya Khare is designing surfaces which can be tuned to achieve desired wetiing behavior by liquids. This group uses patterning by polymers or inorganic materials like titania, or both to control the wetting response \textbackslash cite\{Pant2014777\}. The techniques used and developed by them are relevant to the proposed project. Especially, the proposed project will also use nanopatterned titania surfaces to tune the wetting behavior by using radiation.\subsection{3.4 Sanjay Puri}

\par 3.4.1 Theoretical studies for Phase equilibria near surfaces: J. Chem. Phys. 139, 174705 (2013)
\par 3.4.2 Studies on the effect of temperature gradients on the phase behavior of fluids: Europhysics Letters, 103 (2013) 66003\subsection{3.5 Chinmay's group compares sessile drop and calorimetric approaches}

\par 3.5.1 Useful for justifyng the microcalorimetry
\par 3.5.2 Can coating density affect the temperature dependence?
\par 3.5.3 Can mix of different coatings be tested to "tune" hydrophilic(phobic) nature?\section{4 Novelty of our work}

\par 4.1 The survey of the research field indicates that the measurements or calculations concerned with the thermal response of wetting behavior are very scarce. The available measurements involve solid-fluid systems that are complex in the sense that multiple factors influence their thermal response. In order to rationally design the surfaces, it is first necessary to approximately isolate the different factors and study their roles individually. The first factor that we plan to isolate is the effect of temperature on the arrangement of molecules of the solid. We will only consider the solids whose structure is negligibly affected in the desired temperature range. This will ensure that the thermal response of wetting only depends on the arrangement of fluid molecules near the surface and the nature of solid-fluid interaction. We then divide the features of the surface into three categories: 1) Average solid-fluid interaction strength, 2) surface roughness and 3) chemical heterogeneity of the surface. We will systematically investigate each of these features using experiments and molecular simulations. Molecular simulations will be used for surfaces with nanoscopic or atomic-scale features, whereas experiments will be used to study features at larger scales. The focus will on the determination of equilibrium solid-fluid interfacial properties and their variation with temperature. The collected data will be analyzed to identify trends with respect to different variables related to the solid surface.
\par 4.2 We note that the objective of molecular simulations is not the quantitative comparison with the experimental results, but the identification of trends with respect to the surface features that are difficult to manipulate using experiments. Such a strategy provides us the flexibility in selecting the solid and fluid models to be used in simulations. In the proposed project we will use simple models that enable us to investigate a variety of solid surfaces in a computationally efficient manner. Though these models will be less realistic, they will help provide fundamental insights into the molecular-scale phenomena associated with the temperature-dependent wetting behavior. For example, in real systems, both the electrostatic as well as non-electrostatic interactions between molecules may influence the interfacial phenomena. The individual effect of these interactions can be studied by simulating models of molecules that interact with different proportion of the above types of interactions \textbackslash cite\{rane2014understanding\}. One novelty of our project will be the manner in which the experimental and the computational results will be analyzed. We will combine the "trends" that are obtained from the two sets of results. This will help us to understand the dependence of wetting behavior and its thermal response on the characteristics of a particular surface feature at different scales. Such an approach is different from the one where computational and experimental results are compared for each solid-fluid system.\subsection{4.3 In the proposed project we want to understanding the role of transverse density correlations (TCs) in the thermal response of wetting behavior. This study will involve the following important steps}

\par 4.3.1 Comparing the contribution from TCs with those from other molecular-scale phenomena occuring near solid-liquid interfaces. We will also understand how the relative contributions are affected by the variation of the different surface parameters.
\par 4.3.2 Considering the role of TCs at liquid-vapor interface in the thermal response of the wetting behavior. Here, the emphasis will be on the 1) liquid-vapor interface that is bounded by the nanoscopic structures on a rough solid surface, and 2) liquid-vapor interface that is bounded by a pore of nanoscale dimensions
\par 4.3.3 Considering the implications of the above understanding on the design of the solid surfaces. Here, one important objective will be to study the possibility of designing surfaces to control contact angles and their temperature derivatives almost independent of one another.\subsection{4.4 We are also interested in developing computer simulation based approaches to guide the design of solid surfaces with desired thermal response of the wetting behavior. The important steps will be as follows:}

\par 4.4.1 Defining suitable phase spaces for statistical mechanics-based simulations. The extended phase spaces will consider the different design possibilities of a solid surface at atomistic and nanoscopic scales.\subsubsection{4.4.2 Development and efficient application of free-energy-based approaches to compute interfacial properties. This will involve}

\par 4.4.2.1 Efficient collection of multi-dimensional transition matrix
\par 4.4.2.2 Extracting higher order moments of the probability distributions and using them to gain insights about the molecular-scale phenomena.
\par 4.4.2.3 Expanding the scope of these approaches to use data from different types of statistical-mechanic-based computational methods like Monte Carlo, molecular dynamics, DFT and integral equation theories\subsection{4.5 We are interested in studying the thermal response of porous systems and understanding them in terms on the studies performed on flat surfaces. Experiments will be performed on the absorption of the same fluid used above. The porous systems will be made of some of the same solids and their mixtures. The main steps will be:}

\par 4.5.1 Comparison of the thermal response of system with that estimated from the studies performed on flat surfaces
\par 4.5.2 Estimating the design variables of the packed column based on the information from the studies on flat surfaces and thermodynamic modelling. The objective of the design being to achieve the desired thermal response.\section{5 Work plan}
\subsection{5.1 Collection of data for \$\textbackslash gamma\_\{\textbackslash mathrm\{sv\}\} - \textbackslash gamma\_\{\textbackslash mathrm\{sl\}\}\$ and \$\textbackslash frac\{\textbackslash mathrm\{d\}(\textbackslash gamma\_\{\textbackslash mathrm\{sv\}\} - \textbackslash gamma\_\{\textbackslash mathrm\{sl\}\})\}\{\textbackslash mathrm\{d\}T\}\$. The objective here is to determine the trends in the above properties with respect to the parameters related to the solid surface. Experiments will be used to understand the effect of features that can be manipulated on a larger scale (greater than 100nm). Molecular simulations will be used for features that range between atomistic scale and few nanometers where deviations from the common theoretical models are observed. The focus of molecular simulation studies will be on the variation of the abovementioned properties with the features of the solid surface, instead of quantitative comparison with real systems. Therefore, the choice of models used for simulations and the employed system conditions will depend on the computational feasibility.}

\par 5.1.1 \textbackslash loadfig\{1\} We describe here the information that we plan to obtain from these studies. Let us denote the three surface features by A, B and C. A "feature" may be related to a particular characteristic of the surface like overall solid-fluid interaction or roughness. A particular feature may be manipulated by changing certain characteristics of the surface. For example, roughness may be manipulated by using solid surfaces with different roughness factors. Our first aim is then to collect the data to be represented on the plot of \$\textbackslash frac\{\textbackslash mathrm\{d\}(\textbackslash gamma\_\{\textbackslash mathrm\{sv\}\} - \textbackslash gamma\_\{\textbackslash mathrm\{sl\}\})\}\{\textbackslash mathrm\{d\}T\}\$ vs \$\textbackslash gamma\_\{\textbackslash mathrm\{sv\}\} - \textbackslash gamma\_\{\textbackslash mathrm\{sl\}\}\$ as shown in Figure \textbackslash ref\{repdgvsg\}. In this figure, the points enclosed in the different shaded regions represent the data that is collected by varying the surface features A, B and C. This will include the results from simulations and experiments. For example, if A is surface roughness then molecular simulations will be used to study the effect on roughness at scales smaller than few nanometers whereas experiments will be used for larger scales. The data as represented in the Figure \textbackslash ref\{repdgvsg\} can help us to systematically study the trends with respect to each surface feature. Each of the "features" that we mentioned above are broad categories that can be manipulated by changing different surface characteristics. For example, a rough surface contains  different surface features like hills and valleys (Figure \textbackslash ref\{repsurffeaturesrough\}). The roughness feature can then be manipulated by changing the depth of the valleys or by changing the spacing between hills. The points obtained by changing the former characteristic may lie on a curve that is different from the one obtained by changing the later. This is represented in Figure \textbackslash ref\{repdgvsg\} where the curves \$l\_\{\textbackslash mathrm\{1\}\} and \$l\_\{\textbackslash mathrm\{2\}\}\$ represent the trends obtained by manipulating the feature A in two different ways. We are interested in identifying such trends from the collected data. Though the main goal of this project is to understand the effect of solid surface, we plan to obtain results with few fluids to undderstand the variation with fluid nature. Typically, these fluids will differ in the proportion of electrostatic and dispersion interactions between theire molecules or ions. For example, experiments will be performed using fluids like hexadecane (lacking electrostatic interactions), water (both electrostatic and nonelectrostatic) and room temperature ionic liquids (electrostatic interactions are dominating). Coming to the molecular simulations, we will use fluid models with different proportion of electrostatic and dispersion interactions. These models will mostly contain monoatomic particles interacting with electrostatic and non-electrostatic interactions, the parameters of which will vary for different fluids. The choice of such a simple model of fluid has two advantages: 1) Only the desired type of fluid-fluid interaction is considered, excluding the effects of the shape of the molecule, and 2) the computation of properties is less expensive.  In the following paragraphs we describe the main surface features that we will investigate.
\par 5.1.2 Average interaction between the solid surface and fluid. Here, the role of average interaction between the molecules of fluid and those of the homogeneous surface will be studied. The measurements or computations will be performed using single crystal surfaces with different orientations. The experiments will be performed using single crystal surfaces of compounds like TiO2 and ZnO. The chemical nature of these surfaces will be tuned by using a suitable radiation \textbackslash cite\{wang1999studies\}, and the effect of such modification on the  wetting behavior and its temperature dependence will be studied. Molecular simulations will be performed for the crystalline solid surfaces made of monatomic particles. Here, one can manipulate the interaction between atoms of solid surface and fluid to cover all the wetting regimes, from complete drying to complete wetting.
\par 5.1.3 \textbackslash loadfig\{2\} Roughness of the solid surface. The solid surface can be rough at multiple scales. At the lowest scale, the arrangement of molecules on the face of the crystal is also a form of roughness. In order to study its effect, the studies performed in the previous paragraph will employ solid crystals with different orientation of surface atoms ( say 100, 110 or 001 crystal faces). The effect of large scale roughness will be experimentally studied using TiO2, ZnO and SiO2 surfaces. The roughness will be modulated using plasma \textbackslash cite\{barshilia2014superhydrophobic\} or chemical itching \textbackslash cite\{tuteja2008robust\}. The duration of the itching process and the conditions employed will enable us to control the roughness. Though most of the above surfaces will lack regular geometric features, we plan to study few silical substrates with regular microscopic patterns as shown in Figure \textbackslash ref\{rep\_surffeaturesrough\}. The use of materials like TiO2 and ZnO will also enable us to vary the interaction between the surface atoms and those of the liquid (see previous section). The scale of features that we expect to study in experimental systems will range from 100nm to few micrometers. We will use molecular simulations at the lower scales, where the solid will be crystalline and made of monoatomic molecules. If the solid surface lies in x-y plane, then the surface patterns can be broadly characterized into two types as shown in Figure \textbackslash ref\{repsurffeaturesrough\}: 1) Those that vary along only one axis (stripped pattern) and 2) Those that vary along both x and y axes (chessboard pattern).  We stress that in molecular simulations, the features are always periodic at the macroscopic scale. The shape of these patterns will be rectangular. The studies will be performed for different characteristic dimensions of the above two types of patterns. For example, the rectangular features with different spacings between the columns and different depths of the asperities will be investigated.  As in the previous section, we will also see the effect of varying solid-fluid interaction strengths using advanced simulation techniques.
\par 5.1.4 \textbackslash loadfig\{3\} Chemically heterogeous solid surfaces. We first discuss the surfaces for experimental studies. At the atomistic level, chemical heterogeneity is synonimous with the structure of individual molecules forming the surface and their arrangement in crystal. Hence, the measurements to be performed for different crystal faces of TiO2 (see previous paragraph) will also include the effect of chemical heterogeneity. It is also possible to achieve chemical heterogeneity at similar or slightly larger scales by using surfaces made of block co-polymers or by chemical functionalization. However, we expect that the properties of such solid surfaces will be significantly affected by temperature. Therefore, we will not investigate such heterogeneity in the current project. For scales ranging from 100nm to few micrometers, we will use TiO2 and ZnO surfaces that are made heterogeneous by selective exposure to radiation. Studies will be performed with two types of patterns : 1) Stripped and 2) Chess board (Figure \textbackslash ref\{repsurffeatureschem\}). For each type, we will also consider different dimensions of the features.  We will use molecular simulations to cover the features ranging from atomic scale to few nanometers. The solid surface will be crystalline and made of monoatomic molecules. In order to study the atomic scale heterogeneity, we will use different interaction parameters for atoms  in the liquid-facing part of the unit cell (Figure \textbackslash ref\{repsurffeatureschem\}. On larger scales, we will consider stripped and chessboard patterns with different dimensions of the features. The patterns will contain regions of  two types that are distinguished by the interaction parameters between solid and fluid. The patterns containing regions of more than two types will be considered in future studies.\subsubsection{5.1.5 In this section we describe the methods that will be used to compute \$\textbackslash gamma\_\{\textbackslash mathrm\{sv\}\} - \textbackslash gamma\_\{\textbackslash mathrm\{sl\}\}\$ and \$\textbackslash frac\{\textbackslash mathrm\{d\}(\textbackslash gamma\_\{\textbackslash mathrm\{sv\}\} - \textbackslash gamma\_\{\textbackslash mathrm\{sl\}\})\}\{\textbackslash mathrm\{d\}T\}\$ from molecular simulations. Most of the studies will be conducted by using Metropolis Monte Carlo algorithm via a free-energy based approach. In this approach we compute the interfacial free energies like \$\textbackslash gamma\_\{\textbackslash mathrm\{sv\}\} - \textbackslash gamma\_\{\textbackslash mathrm\{sl\}\}\$ and \$\textbackslash gamma\_\{\textbackslash mathrm\{lv\}\}\$ by using the probability distributions that are collected from the  simulations. These distributions are collected as a function of suitable order parameter like number of fluid molecules or solid-fluid interaction strength \$\textbackslash varepsilon\_\{\textbackslash mathrm\{sw\}\}\$. The system that is simulated is shown in the snapshots accompanying the Figure \textbackslash ref\{repintpotential\}. The simulation cell is parallelopiped and is bounded by the solid surface of interest on the lower side and a control wall on the upper side. The nature of the control-wall is described below. The simulation is performed such that  cell is periodic in the x and y directions along the plane of the solid surface. Note that the positions of the molecules of the surface are fixed during the simulation because we will only study the "rigid" surfaces.  We will work in the grand canonical (GC) ensemble where we fix the temperature T, volume of the cell V and the chemical potential of the fluid \$\textbackslash mu\_\{\textbackslash mathrm\{sat\}\}\$ that corresponds to the liquid-vapor coexistence conditions. When the number of molecules is less, the configuration corresponds to the vapor in contact with the surface. Large number corresponds to a liquid slab attached to one of the two surfaces. The simulations may be classified into two types depending on the order parameter used to collect the free energies: 1) Direct simulations and 2) Expanded ensemble simulations.}

\par 5.1.5.1 \textbackslash loadfig\{4\} Direct simulations \textbackslash cite\{rane2011monte\}. Here the main order parameter is the number of molecules of the fluid. These may be further divided into two types depending on the nature of the control wall. The control wall is an external field that acts over a very small range and is not atomistically detailed.  Typically, its influence extends to only few molecular diameters away from it. If the interaction strength of this wall with the fluid \$\textbackslash varepsilon\_\{\textbackslash mathrm\{cw\}\}\$ is much greater than that of the solid surface (\$\textbackslash varepsilon\_\{\textbackslash mathrm\{sw\}\}\$), then we refer to such simulations as drying simulations. On the other hand if \$\textbackslash varepsilon\_\{\textbackslash mathrm\{cw\}\} << \textbackslash varepsilon\_\{\textbackslash mathrm\{sw\}\}\$, then the simulations are called spreading simulations. In the former case, the liquid slab "grows" on the control wall and in the later case, it grows on the solid surface of interest. The probability distribution collected from this simulations is used to compute the interface potential \$V\$ as shown in the Figure \textbackslash ref\{repintpotential\}. The different parts of the profile are proportioanal to interfacial free energies. For spreading simulation: \textbackslash loadeq\{1\} \textbackslash loadeq\{2\} Here, \$A\$ is the cross-setional area of the simulation cell and \$C\_\{\textbackslash mathrm\{spread\}\}\$ is a constant depending on the cell geometry.  For drying simulations: \textbackslash loadeq\{3\} \textbackslash loadeq\{4\} Both these simulations are performed for a a particular surface and \$\textbackslash gamma\_\{\textbackslash mathrm\{sv\}\} - \textbackslash gamma\_\{\textbackslash mathrm\{sl\}\}\$ and \$\textbackslash gamma\_\{\textbackslash mathrm\{lv\}\}\$ are obtained by the solving the above system of equations. Once \$\textbackslash gamma\_\{\textbackslash mathrm\{lv\}\}\$ at a particular temperature is known, one can then proceed with only one of the abovementioned simulations to calculate solid-fluid interfacial properties for other surfaces. Spreading simulations are preferred for solid surfaces with stronger attraction towards fluid molecules, whereas drying simulations are used for weaker surfaces. However, performing direct simulations on different surefaces is computationally expensive, especially since we want to scan a broad range of surface characteristics.
\par 5.1.5.2 Expanded ensemble simulations \textbackslash cite\{rane2011monte\}. In the plots from direct simulations, one may notice that the regions of interest are the valleys and plateaus. Therefore, if one has the direct simulation results for a particular surface, then one only needs to track the position of the valley and the plateau with a desired surface characteristic. This is acheived using the expanded ensemble (EE) simulations. First, an order parameter, say \$x\$, is defined which characterizes the surface feature to be studied. This can be \$\textbackslash varepsilon\_\{\textbackslash mathrm\{sw\}\}\$ or a parameter related to the roughness or the chemical heterogeneity of the surface. One then performs EE simulations where, along with the number of molecules \$N\$, even x is allowed to change and the probability distribution of \$x\$ is used to calculate the relative free energies. Unlike the direct simulations, \$N\$ is allowed to vary in a small region, like the valley or a small part of the plateau in the Figure \textbackslash ref\{repintpotential\}. The process is depicted by dashed lines in the Figure \textbackslash ref\{repintpotential\}. There are four different types of simulations that can be performed. As in the direct simulations, \$V\_\{\textbackslash mathrm\{valley\}\}\^\{S\}\$ and \$V\_\{\textbackslash mathrm\{plateau\}\}\^\{S\}\$ are tracked for stronger surfaces, whereas the drying counterparts are tracked for weaker surfaces. This strategy enables one to scan a large parameter space with less computational expense. Another important advantage comes from the fact that the interfacial free energies are calculated at the closely space values of the desired order parameter. This is useful to numerically compute the quantities like \$\textbackslash frac\{\textbackslash mathrm\{d\}(\textbackslash gamma\_\{\textbackslash mathrm\{sv\}\} - \textbackslash gamma\_\{\textbackslash mathrm\{sl\}\})\}\{\textbackslash mathrm\{d\}\textbackslash epsilon\_\{\textbackslash mathrm\{sw\}\}) which can provide important information about the nature of the wetting or drying transition.
\par 5.1.5.3 The calculation of \$\textbackslash frac\{\textbackslash mathrm\{d\}(\textbackslash gamma\_\{\textbackslash mathrm\{sv\}\} - \textbackslash gamma\_\{\textbackslash mathrm\{sl\}\})\}\{\textbackslash mathrm\{d\}T\}\$. One straightforward way of calculating this quantity is to perform the direct and expanded ensemble (EE) simulations at two adjacent temperatures. In addition to being computationally expensive, this strategy suffers from the challenge of selecting suitable adjacent temperature. If the adjacent temperature is too close, then the precision of the above quantity will be poor. On the other hand if it is too far, the numerical derivative will not be accurate. Another alternative is to use EE simulations with order parameter as temperature. However, such simulations can only be performed for a particular solid surface, and will not be practical for the current project because of the variety of surfaces that we wish to investigate. In the proposed project we will employ the Gibbs-formalism-based approach to extract the temperature-derivatives of interfacial free energies from a simulation data performed at single temperature. Following from interfacial equivalent of Gibbs-Duhem equation, one can write: \textbackslash loadeq\{5\} Here, \$s\_\{\textbackslash mathrm\{sv\}\} (s\_\{\textbackslash mathrm\{sl\}\})\$ is the solid-vapor (solid-liquid) excess entropy per unit area of the interface and \$\textbackslash Gamma\_\{\textbackslash mathrm\{sv\}\} (\textbackslash Gamma\_\{\textbackslash mathrm\{sl\}\})\$ are the excess number of molecules per unit area at the solid-vapor (solid-liquid) interface. Any excess quantity \$X\_\{\textbackslash mathrm\{sf\}\}\$ for the interface between solid and a fluid (liquid or vapor) is given as follows: \textbackslash loadeq\{6\} Here, \$X\_\{\textbackslash mathrm\{total\}\}\$ is the quantity corresponding to the total system and \$X\_\{\textbackslash mathrm\{f\}\}\$ is the quantity of fluid in absence of the interface. Among the quantities in the RHS of Equation (\textbackslash ref\{eqn5\}), \$\textbackslash Gamma\_\{\textbackslash mathrm\{sv\}\} - \textbackslash Gamma\_\{\textbackslash mathrm\{sl\}\}\$ can be calculated from the probability distributions of the number of molecules in the EE simulations. In order to compute \$\textbackslash frac\{\textbackslash mathrm\{d\}\textbackslash mu\_\{\textbackslash mathrm\{sat\}\}\}\{\textbackslash mathrm\{d\}T\}\$, we use a form of Clausius-Clapeyron equation \textbackslash cite\{frenkel2001understanding\}: \textbackslash loadeq\{7\} Here \$\textbackslash rho\_\{\textbackslash mathrm\{sat\}\}\^\{\textbackslash mathrm\{f\}\}\$ denotes the number of molecules of the bulk fluid \$\textbackslash mathrm\{f\}  = \textbackslash mathrm\{l\} (liquid) or \textbackslash mathrm\{v\} (vapor)\$ in a unit volume, under the liquid-vapor saturation conditions. \$u\_\{\textbackslash mathrm\{sat\}\}\^\{\textbackslash mathrm\{f\}\}\$ denotes the bulk energy density for a fluid under liquid-vapor saturation conditions. The excess entropies can be computed from the molecular simulations via the following expression: \textbackslash loadeq\{8\} Here \$e\_\{\textbackslash mathrm\{sf\}\}\$ denotes the excess energy of the solid-fluid interface (\$\textbackslash mathrm\{f = l\}\$ for liquid or \$v\$ for vapor). It is obtained using the ensemble average of interaction energy between the molecules of fluid and that between fluid and solid surface. This approach enables us to compute \$\textbackslash frac\{\textbackslash mathrm\{d\}(\textbackslash gamma\_\{\textbackslash mathrm\{sv\}\} - \textbackslash gamma\_\{\textbackslash mathrm\{sl\}\})\}\{\textbackslash mathrm\{d\}T\}\$ over a wide variety of solid surfaces. Finally, we note that \$\textbackslash frac\{\textbackslash mathr\{d\}\textbackslash gamma\_\{\textbackslash mathrm\{lv\}\}\}\{\textbackslash mathrm\{d\}T\}\$ can also be obtained from our simulations using the similar strategy. The calculation of temperature derivative of cosine of contact angle \$\textbackslash mathrm\{cos\}\textbackslash theta\$ can be then be straightforwardly acheived via the Young's equation. as follows: \textbackslash loadeq\{9\}\subsubsection{Experimental procedures.}

\par In later section we describe the specific tests that will be used to ensure the rigidity of these surfaces. [Advances in Colloids and Interface Science, 227, 57 (2016)].
\par 5.2 Time schedule
\par 5.3 Use of research outcome
\par 5.4 Risk analysis\section{6 Expertise and preliminary results}
\subsection{6.1 Computational}

\par 6.1.1 For a ghost solute, how ensemble averages of Equation (25) of cap\_wave\_field1 paper change with temperature for bulk liquid and interface
\par 6.1.2 How the contribution from capillary waves (deldelmu) affected by temperature?
\par 6.1.3 Studies with 2 solutes with different separation at bulk and interface: 1) How coeffs of Equation (25) ar affected by distance and 2) How they are affected by temperature
\par 6.1.4 Temperature dependence of wetting behavior on a lattice surface different than the one earlier observed\subsection{6.2 Experimental}

\par 6.2.1 measuring contact angles and temperature dependence on flat surfaces
\par 6.2.2 Measurin wetting properties in porous beds
\par 6.2.3 Analysis of crystal faces and working with crystals using processes like milling\section{7 Money and technical resources}
