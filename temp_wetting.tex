
\documentclass[a4paper,12pt,single,pdftex]{scrartcl}
\usepackage{ngerman}
 \usepackage{color}  
 \usepackage{html}  
 \usepackage{times}  
 \usepackage{graphicx} 
 \usepackage{fancyheadings}  
 \usepackage{hyperref}  
 \setlength{\parindent}{0.6pt} 
 \setlength{\parskip}{0.6pt} 
 \title{Designing solid surfaces with desired temperature dependence of wetting behavior}
 

\begin{document} 
\maketitle
\newpage

\label{ID_1723255651}\label{ID_268635931}\section{Scientific rationale}

\label{ID_1490725096}\subsection{The interaction between a solid surface and a liquid plays an important role in several natural and synthetic processes. An important phenomena in such systems is the wetting of a solid surface by liquid. The technological advances have enabled the development and application of solid-liquid interfaces  of increasingly microscopic dimensions. Further, there is a great scope for fine-tuning the chemical nature of solid surface at the molecular scale to achieve desired wetting behavior. Theoretical studies and computer simulations can help guide such manipulation. In this particular project we focus on the temperature dependence of the solid-liquid interfacial properties. Specifically, we will employ statistical mechanics to design solid surfaces that show desired thermal response of wetting. This will be achieved with the help of advanced molecular simulation techniques.}

\label{ID_1521482540}\subsection{Broadly speaking, the temperature dependence of solid-liquid interfacial properties are important where the solid-liquid interface plays an important role. In microfluidics and nanofluidics wetting behavior is very important and there are numerous scenarios where such devices encounter environments with varying temperatures. For example, microfluidic devices used in processes involving endothermic or exothermic reactions. Another important application is the role of temperature in the wetting of ink droplets on solid surfaces. With the application of inkjet printing technology in areas like microelectronics and solarcells it becomes important to understand how the wetting behavior of ink droplets epends on temperature. The knowledge about dynamics of heat and mass transfer is indeed crucial in the development of such realistic systems. However, it is useful to first study such systems under equilibrium conditions to understand the factors that must be considered for designing them. Moreover, such a study can provide useul insights into the underlying molecular-scale phenomena. We explain this point with the example of thermal engines based on the wetting behavior of solids [AIChE, 49, 764 (2003)]. These engines are based on a thermodynamic cycle made of processes that involve the spreading of liquid on a solid surface and have applications in micro-mechanical actuators [Energy and Environmental Science, 4, 3632] and car dampers [J. of Automobile Engg., 221, pp. 301]. If one assumes that all the involved processes are reversible then it is straightforward to relate the technologically relevant quantities of the engine to molecular-level details of the system. For example, the exchange of work and heat with the surroundings is related to the interfacial free energies (solid-liquid, liquid-vapor and solid-vapor) and their derivatives with respect to temperature, respectively. Using the framework of statistical mechanics these quantities can be further related to the chemical nature of solid surface and liquid. Thus, by designing solid surfaces that result in the desired thermal response of the wetting behavior one can, in principle, develop wetting engines having desired efficiency.}

\label{ID_412834846}\subsection{With regards to the nature of solid surface, there are several factors that can potentially affect solid-fluid interfacial free energiesand their temperature derivatives. In this project we are partiularly interested in the effect of heterogenity along the plane of the solid surface. Such a heterogenity can be due to the variation of the chemical nature or due to the surface roughness. Moreover,  it can range over multiple scales. For example, roughness can be considered at the atomistic level due to a particular arrangement of surface atoms, whereas at larger scales it can be due to variation in the macroscopic topography of the surface. Such heterogeneity affects the solid-fluid interfacial properties by influencing the correlations between local densities of the fluid molecules along the plane parallel to the surface. Therefore, in order to rationally design such a heterogeneous surface it is important to understand the contribution of abovementioned correlations to solid-fluid interfacial free energies and their derivatives with respect to temperature. In this document we refer the density-density correlations in liquid along liquid-vapor or solid-liquid interfaces as transevrse correlations (abbreviated as TC)}

\label{ID_350936959}\subsection{The important objectives of this project can be summarised as follows:}

\label{ID_1671481601}\subsubsection{Systematic study of the equilibrium thermal response of wetting on the rigid solid surfaces using experiments, thermodynamic models for macroscopic interfaces and molecular simulations. The properties of cos theta and its temperature derivatives will be calculated for the same fluid on different surfaces. Experiments will be used for real systems. Calculations based on thermodynamic models for macroscopic wetting behavior will be used for ideal surfaces with features on microscopic scale. Molecular simulations and Statistical mechanics based theories will be used for calculations on surfaces with nanoscpic or atomic-scale features. The results from different studies will be combined to understand the trends with respect to the following system parameters.}

\label{ID_609905923}\paragraph{Effect of the surface-fluid interaction averaged over the surface}

\label{ID_589844027}\paragraph{Effect of nanoscopic and microscopic roughness}

\label{ID_1898685670}\paragraph{Effect of chemical heterogeneities along the solid surface at atomistic, nanoscopic and microscopic cales}

\label{ID_1074266462}\subsubsection{Studying the thermal response of porous systems and understanding them in terms on the studies performed on flat surfaces. Experiments will be performed on the absorption of the same fluid used above. The porous systems will be made of some of the same solids and their mixtures. The main objectives will be}

\label{ID_725380065}\paragraph{Comparison of the thermal response of system with that estimated from the studies performed on flat surfaces}

\label{ID_322656915}\paragraph{Estimating the design variables of the packed column based on the information from the studies on flat surfaces and thermodynamic modelling. The objective of the design being estimating the desired thermal response}

\label{ID_1058727089}\subsubsection{Developing computer simulation based approaches to guide the design of solid surfaces with desired wetting properties. The focus will be on the equilibrium thermal response of the wetting behavior. The main objectives will be as follows:}

\label{ID_957715212}\paragraph{Defining suitable phase spaces for statistical mechanics-based simulations. The extended phase spaces will consider the different design possibilities of a solid surface at atomistic and nanoscopic scales.}

\label{ID_534544736}\paragraph{Development and efficient application of free-energy-based approaches to compute interfacial properties. This will involve}

\begin{itemize}
\label{ID_1978254381}\item Efficient collection of multi-dimensional transition matrix
\label{ID_1273631572}\item Extraching higher order moments of the probability distributions and using them to gain insights about the molecular-scale phenomena.
\label{ID_916433473}\item Expanding the scope of these approaches to use data from different types of statistical-mechanic-based computational methods like Monte Carlo, molecular dynamics, DFT and integral equation theories
\end{itemize}
\label{ID_1978254381}\label{ID_1273631572}\label{ID_916433473}\label{ID_1343604658}\subsubsection{Understanding the role of transverse density correlations (TDCs) in the thermal response of wetting behavior. This study will involve the following important steps}

\label{ID_1458428234}\paragraph{Comparing the contribution from TDCs at solid-liquid interface with those from other molecular-scale phenomena occuring near solid-liquid interfaces. We will also understand how the relative contributions are affected by the variation of the different surface parameters.}

\label{ID_1578421438}\paragraph{Considering the role of TDCs at liquid-vapor interface in the thermal response of the wetting behavior. Here, the emphasis will be on the 1) l-v interface that is bounded by the nanoscopic structures on a rough solid surface. Say, in the Cassie regime. 2) l-v interface that is bounded by the boundaries of a small pore of nanoscale dimensions}

\label{ID_703377776}\paragraph{Considering the implications of the above understanding on the design of the solid surfaces. Here, one important objective will be to study the possibility of designing surfaces to control contact angles and their temperature derivatives almost independent of one another.}

\label{ID_1110611260}\section{Background}

\label{ID_1392626561}\subsection{Research status (international)}

\label{ID_688891944}\subsubsection{Theoretical}

\label{ID_1917907773}\paragraph{The temperature dependence of wetting phenomenon has been closely connected with the study of wetting  from the perspective of phase coexistence. Basically, one can consider the two scenarios: 1) a solid-vapor interface being partially wet by the liquid and 2) solid-vapor interface being completely wet by the liquid as two surface "phases". Similar to coexistence between say liquid and vapor these two surface phases can be in coexistence at certain system parametrers. The parameters of interest are temperature or those related to the nature of the solid surface like the strength of interaction between solid and liquid, surface roughness etc. Analogous to wetting transitions, one may consider drying transitions where the role of liquid and vapor in the above tweo scfenarios are interchanged. [The adjacent figure 1 denotes the wetting and drying transitions]}

\label{ID_1762818689}\paragraph{In1977 Cahn proposed that there exists a temperature below the critical temperature of fluid at which the wetting transition on a given solid surface occurs. The argument was based on the comparison between the temperature dependence of \gamma_{sv} - \gamma_{sl}  and that of \gamma_{lv}. Cahn used a theoretical approach called as square-gradient version of Van der Waals theory and did not consider the atrraction between solid and liquid in explicit detail.}

\label{ID_1622008046}\paragraph{Later, the square-gradient theory was modified by Nakanishi and Fisher to include the dependence of solid-fluid attraction. Specifically, Nakanishi and Fisher proposed a phase diagram which represented the coexistence lines for the wetting and drying transitions in the plot of temperature vs a parameter denoting the strength of solid-fluid interaction. Their phase diagram shows that for weak solid-liquid interaction, increase in temperature brings the system closer to drying transition. Whereas for strong solid-fluid interactions, increase in temperature brings the system closer to the wetting transition. It should be noted, however, that  these theories do not consider the molecualr-level details of the solid surface or liquid. Therefore, the observations do not provide quantitative predictions of the wetting behavior in real systems.. Secondly, the above studies only understand the behavior of systems "at" the wetting and drying transitions. From a technological perspective, it is also important to understand the temperature dependence of contact angles in the partial wetting regime.}

\label{ID_1378614173}\paragraph{Another theoretical approach that is applied to understand the wetting behavior is claasical Density Functional theory (DFT). This theory considers molecular-level details of the system but is computationally less expensive as compared to molecular simulations. It considers the system in grand canonical ensemble and expresses the grand potential as a functional of the space-dependent density of the fluid. The equilibrium density of the fluid is calculated by minimizing the grand potential. The interfacial free energies like \gamma_{sv} \gamma_{sl} and \gamma_{lv} are then obtained from the equilibrium density profile via the graand potential. Though this approach can be only used for very simple systems, it is able to provide qualitative understanding of the observations in real systems}

\label{ID_53484468}\paragraph{One of the first applications of DFT to study wetting behavior is by Sullivan (J. Chem Phys, 74, 2601) who employed the Van der Vaals theory without the square-gradient approximation. Particularly, the Cahn's definition of solid surface was reinterpreted in terms of a more realistic solid-fluid interaction potential. This enabled Sullivan to study the variation of contact angle theta over a range of solid-fluid interaction strengths (\epsilo_w ). at multiple temperatures. It was observed that for small \epsilon_{w} - where theta is large - \theta increased with temperature. On the other hand, for large \epsilon_{w} \theta decreased with temperature. Similar influence of solid-fluid interactions on the temperature dependence of \theta was also observed in more recent DFT studies. Kuiper and Blokhuis (J. Chem. Phys, 131, 044702 (2009)) used DFT to reinterpret the square-gradient model of Nakanishi and Fisher described earlier in terms of the realistic solid-fluid interactions. In addition to verifying the phase diagram of wetting and drying transitions as observed by Nakanishi and Fisher, they computed the contact angles for a wide range of substrate-fluid interaction strengths \epsilon_{w}. Here, \epsilon_{w} was characterized by the depth of square-well potential between solid and fluid. As in study of Sullivan, contact angles increased with temperature for small \epsilon_{w} whereas, opposite trnd was observed for large \epsilo_{w}. Recently, Berim and Ruckenstein employed the canonical version of DFT to study the temperature dependence of contact angles over a range of substrate-fluid interaction strengths. The system they studied was a liquid drop on a solid surface instead of a liquid film in contact with surface as studied by earlier mentioned works. The contact angle was also measured visually at the three phase contact line. Their solid surface was more realistic than the earlier studies and consisted of discrete molecules interacting with fluid molecules via Lennard Jones potential. The LJ parameter for the solid-fluid interaction \epsilon_{sf} characterized the substrate-fluid interaction strength. Like earlier studies, they observed that \theta increased with temperature for low \epsilon_{sf} whereas opposite trends were observed for larg \epsilo_{sf}. The notable distinction from the earlier studies was that the transition from the former regime to the later regime happened over a very narrow range of \epsilo_{sf}. Berim and Ruckenstein commented on the significance of the intermediate substrate-strength \epsilon_{sf}^o at which the \theta is almost independent of temperature. The above studies indicate that the temperature dependence of \theta show some general trends with respect to the strength of attraction between surface and fluid, irrespective of the actual form of interaction potential. That being said, the complexity of systems that can be handled using DFT is limited. Especially, the heterogeneity of the solid surface that may play an important role in the wetting behavior was not considered.}

\label{ID_678529580}\paragraph{We finally note that the observed phase diagrams of wetting-drying transitions and the influence of solid-fluid attraction strength are related. The latter behavior may be roughly predicted from the phase coexistence curves in the T vs \epsilon_{w} plane as shown in [Figure 2]. One may observe that as T increases, the range of \epsilon_{w} corresponding to the partial wetting regime narrows down and therefore the plots of cos\theta vs \epsilon_{w}become steaper as recently commented by Henderson [Discussion notes on European Physical Journal, 197, 163 (2011)]. However, it is not trivial to explain the vaery narrow range of \epsilon_{w} over which the inversion in T-dependence occurs. Another important point is that, apart from the strength of the surface-fluid interaction, very few theoretical studies have attempted to understand the temperature dependence in terms of the role played by the molecular scale phenomena. Most of the arguments invoke the macroscopic physics of the system, say related to the surface phase coexistence. Though it is just a matter of interpretation, explaining the observations in terms of molecular scle phenomena can aid in the technological application of various observations.}

\label{ID_869146050}\subsubsection{Experimental}

\label{ID_220869422}\paragraph{Wetting of liquids on solid surfaces is a vast field with numerous applications. In recent years, numerous experimental studies have documented wetting properties on natural and artificial surfaces. This trend is fuelled by the improved ability to manipulate microscopic and nanoscopic details of the surfaces and the measurement devices. One of the most important sub-field of the wetting research is the investigation of solid surfaces that have superhydrophobic properties. In most studies, the equilibrium property that is measured is the contact angle that a small liquid drop makes with the solid surface. In others, instead of liquid drop the force of adhesion between liquid and a solid surface is measured, or the time required for absorption of a liquid in a column packed with the desired solid material is determined. The later two methods then use theoretical models to relate the measured quantities to the contact angle. Most of these studies use contact angle to characterize the hydrophobicity or hydrophilicity of the solid surfaces. Generally, the main goal of such studies is to design a solid surface by manipulating a single feature of solid surface, either roughness  [Nanoscale, 8, 4635 (2016)] or soldi-fluid interaction [J. Phys. Chem. B (1999), 103, 2188-2194]. Relatively fewer studies have been performed to understand the wetting behavior where combinations of two or more surface features are considered [Chem. Rev., 2014, 114 (19), pp 10044–10094] Coming to the temperature dependence, the available experimental data is even rare. The most prominent set of data is by Neumann and colleagues [Journal of Colloid and Interface Science, 4, 105 (1974)],[Advances in Colloid and Interface Science (1974), 4, (2-3), 105-91] and [Journal of Apllied Polymer Science, 42, 1959 (1991)], who studied the temperature dependence of contact angles on different polymer surfaces. It was observed that the contact angles may increase or decrease with the nature of the solid surface, especially the strength of interaction between solid and fluid. These results approximately complied with the expectations from the theories. However, it should be noted that the systems used in these experiments are more complex than those generally used in theoretical investigations. The complex nature of the solid surface meant that many factors related to the surface may have contributed to the observed temperature dependence. For example, in addition to changes on the liquid side of the solid-liquid interface, polymer surfaces may have been significantly influenced by change in temperature.  Recently, there have been studies concerned with the development of solid surfaces which can switch from hydrophilic to hydrophobic behavior with increase in temperature [1) Langmuir, 2003, 19 (7), pp 2545–2549 2) ACS Appl. Mater. Interfaces, 2014, 6 (24), pp 22666–22672  3) Polymer (2014), 55(25), 6552-6560]. These surfaces are made by grafting certain polymers on a solid base. The grafted polymers undergo transition from the extended state to the collapsed state on increasing temperature, which changes their interaction with the liquid molecules, therby resulting in the change of wetting behavior. Thus, the observed thermal response of wetting is actually the manifestation of the phase behavior of polymers in a particular liquid. In the proposed project we plan to first decopule the thermally induced changes occuring in the solid surface from those in the liquid. We will only study solid surfaces whose structure is negligibly influenced by the changes in temperature or by the liquid. Therefore, we don consider systems like solid surfaces grafted with thermal responsive polymers. Such systems will be considered in our In a future study where we plan to gradually make the solid surfaces flexible so that they respond to the changes in temperature.}

\label{ID_566286120}\paragraph{Apart from measurement of contact angles, the wetting behavior on solid surfaces have also been studied by looking at the growth of a thin liquid film on the surface of interest under controlled conditions. This approach is more in line with the theoretical studies concerned with wetting and drying transistions. Typically, the eqilibrium thickness of a liquid film on a surface indicates whether the system is in complete drying (\theta = \pi), partial wetting or complete wetting (\theta = 0) regime. The transitions between each of these regimes can be considered as surface phase transitions akin to those between bulk phases, say liquid and vapor. It is then possible to represent these regimes in a surface phase diagrams in the plane of temperature vs some order parameter characterizing the surface feature of interest. This order parameter may be related to attraction between surface and fluid molecules \epsilon_{w} or roughness   Several such studies have been performed with goal of studying the fundamental aspects of wetting transitions [Rep. Prog. Phys. 64 (2001) 1085–1163]. Most of these studies consider the temperature temperature dependence because it is the system variable that is used to study the transition. However, due to the technical challenges involved in performing such studies, the solid surfaces they investigate are relatively less complex than those employed in the experiments measuring contact angles. As a result, to our knowledge, no systematic data is available as a function of either \epsilon_{w} or roughness. Unlike wetting transitions, drying transitions have not been studied using experiments. Such studies can be useful in understanding the thermal response of wetting on solid surfaces where contact angles are large. That being siad, in the proposed project we do not plan to experimentally study the wetting and drying transitions using the above approach. This is because at this stage, there are several technical challenges in employing these methods to study the solid-fluid systems that we wish to consider. Moreover, it is impractical to employ such techniques to study wetting on many different types of solid surfaces. Finally, we note that unlike solid surfaces, the technical challenges in using the "growth of liquid film approach"  to study wetting on liquid-vapor or liquid-liquid interfaces are less severe. However, such systems are outside the scope of the current project.}

\label{ID_1521593594}\subsubsection{Computational}

\label{ID_836925790}\paragraph{Here we focus our discussion on the Molecular Dynamics (MD) or Monte Carlo (MC) simulation studies performed to study equilibrium wetting behavior of liquids on solid surfaces. The number of such studies have grown in previous few years to predict the wetting behavior on novel surfaces that can be manipulated at increasingly smaller scales. The trend has also been fuelled by the discovery of new applications for the novel substances like room temperature ionic liquids (RTILs) and graphene. The molecular simulations provide two main advantages: 1) They can be used to estimate the wetting behavior in systems that are yet to developed and therefore, guide the rational design of soldi surfaces. This is important because as the scale of features on a surface decreases, the conventional theories that are mostly based on continuum approximation fail because of discreet nature of molecules.  2) They can be used to understand the molecular-level aspects of the wetting behavior that is observed in experiments. The important challenges in any molecular simulation studye are as follows: 1) Upper limit on the system size due to steep increase in the required computational resources. Therefore, as will be done in the proposed study, they are only limited to the systems with sizes less than few nanometers. 2) The results depend on the models of molecules. Ideally, it is expected that the models are parameterized to reproduce the experimental properties. However, if prediction of properties is the objective, such parameterization is not possible. In the current study we do not focus on the the quantitative prediction of properties for real systems. Also, we do not aim to rigorously understand the molecular level phenomena of real systems that are used in experiments. Our objective is to gather the general trends in the macroscopic properties by varying certain features of the solid surface. This allows us to use simple models for molecules of liquids and solids and work at conditions that enable effeicient calculation of desired properties.}

\label{ID_1690050024}\paragraph{Work from Errington group}

\begin{itemize}
\label{ID_473109498}\item Systems that were studied
\label{ID_1487682070}\item How was it explained
\end{itemize}
\label{ID_473109498}\label{ID_1487682070}\label{ID_980454220}\paragraph{Work from Horsch group}

\label{ID_1942058277}\subsubsection{Role of TCs}

\label{ID_896680657}\paragraph{The role of TCs in the temperature dependence of the wetting behavior can be understood in two ways. First, by looking at the contribution of CWs to the interfacial free energies at liquid-vapor-like interfaces. CWs are the undulations of the liquid-vapor interface in the direction perpendicular to the macroscopic interface which result in long-ranged TCs along the l-v interface. Let us say that our system is 3D with l-v interface in x-y plane, then the l-v interface for a particular confguration of molecules can be defined as z = \xi (x,y). Let z = 0 then represent the averaged l-v interface . The undulations of this surface is then analyzed by performing the spectral analysis of \xi as follows: Here, r is the vector  in the x-y plane. As k increases, the corresponding terms denote the contributions from the undulations of short wavelength. The capillary wave theory, defines the CWs as the undulations of very large wavelength, that is k \rightarrow 0. The macroscopic surface tension  \sigma can be expressed as the sum of the "bulk " surface tension  \sigma_b and the one coming from the CWs \sigma_c. [Refer Phys. Rev. A, 33, 1948 (1986)] It is expected that \sigma_c &lt; 0 because undulations are entropically favorable and they decrease the interfacial free energy.  It is also possible to express a \lambda-dependent surface tension \sigma_{\lambda} which excludes the contributions from undulations above a particular wavelenght \lambda}

\label{ID_49666523}\paragraph{Kayser theoretically analyzed \sigma_{\lambda} for temperatures near triple point of liquids and those near critical point. He observed that near the triple point, when \lambda is about 10 times the molecular diameter, \sigma_{\lambda} is about 15% higher than the macroscopic value. For temperatures near Tc, the similar increase of 15% is obtained if \lambda is 20 times the molecular diameter. The contributions from undulations of larger wavelength were also found to be important in different experimental studies [ Nature, 403, 871 (2000) ]. Here, \sigma_{\lambda} was computed from the scattering intensity obtained from a carefully performed X-ray diffraction study of the l-v interface. A more thorough analysis of the temperature dependence of \sigma_{\lambda} was recently performed by Hoefling and Dietrich using molecular simulations. [EPL, 109, 46002]. They observed that the ratio \sigma_{\lambda}/\sigma is significantly altered with temperature. There observations have also been supported by the theoretical study performed by Parry, Rascon and Evans.}

\label{ID_6364435}\paragraph{ Excluding wavelengths greater than \lambda is same as restricting or pinning the l-v interface at distances separated by \lambda along xy plane. (See Adjacent figure). The conclusions from the above studies then imply that surface tension of such a pinned interface shows a different temperature dependence than that of the unpinned or free interface. In real systems, such a pinned interface can be realized in several scenarios like liquid-vapor interface confined inside a nanopore or the interface pinned between the colloidal particles. With regards to wetting, such a scenario also exists on certain solid surfaces. For example, several designs of superhydrophobic or superoleophobic surfaces have protruding stuctures having dimensions between micrometers and nanometers. When liquid sits at the top of these pillars (Cassie regime) a pinned liquid-vapor interface exists between the pillars (See the adjacent figure). Note that the macroscopic solid-liquid interface now consists of the combination of several such pinned liquid-vapor interfaces. The theoretical predictions about the temperature dependence of \sigma_{\lambda} can therefore guide one to tune the distance between these pillars and their arrangement in two dimensions to obtain the desired thermal response of the wetting behavior.}

\label{ID_1590536878}\paragraph{For more general solid surfaces which do not have such structures, the above analysis cannot be used because there are no liquid-vapor-like interfaces near the solid surface. Also for the solid surfaces with pillars, it is only valid when the distance between the pillars is much larger than the molecular diameter of the liquid. Additionaly note that the analysis will be very complicated for the solid surfaces with flexible structures. If the attraction between molecules of solid surface and those of liquid is weak then long-ranged TCs do exist near solid surface. This was recently shown for water molecules near hydrophobic surfaces using molecular simulations [PRL, 115, 016103 (2015)]. In this work Wilding and Evans used quantity called local compressibility \chi as a function of the distance z from the solid surface. The large magnitude of \chi (z) indicates the presence of long-ranged TCs. They observed a distinct peak near surfaces, even for those with contact angles as less as 26^{o}. These results show that TCs can contribute to the solid-liquid interfacial properties. In the paragraphs below we explain how TCs may affect the temperature dependence of  s-l interfacial free energies.}

\label{ID_908289338}\paragraph{Write about studies that looked at the role of TDCs: 1) Velasco and Tarazona, J Chem Phys, 91,7916 (1989) 2) Oleinikova and Brovchenko, Phys. Rev. E, 76, 041603 (2007) and 3) Pleimling, Journal of Physics A, Mathematics and General, 37, R79 (2004)}

\label{ID_1933251243}\paragraph{TDCs have not been experimentally studied. However, experiments have detected the depletion of liquid density near weak solid surfaces. From simulation and theoretical studies, such depletion generally accompanies the increased local compressibility.  However, the actual range of TDCs via experiments remains to be understood. Drying transition not been detected using Experiments. However, the depletion of liquid density near the solid surfaces of very weak strengths has been observed. Refer 1) Hess et al., Phys. Rev. Lett., 78, 1739 (1997) where the depletion of Ne liquid density on Cs surface was onbserved 2)Jensen et al. Phys. Rev. Lett., 90, 086101 (2003) also observed similar depletion of water near hydrophobic surface with Xray reflectiveity studies. They also correlated their observations with contact angle measurements 3) Langmuir, 23, 598 (2007) used neutron diffraction to observe similar depletion between water and hydrophobic solid surface. 4) Poynar et al. Phys. Rev. Lett.,97, 266101 (2006) using x-ray reflectivity measurements}

\label{ID_3142565}\paragraph{The depletion of liquid density near the solid surfaces of very weak strengths has been observed in experiments. Refer 1) Hess et al., Phys. Rev. Lett., 78, 1739 (1997) where the depletion of Ne liquid density on Cs surface was onbserved 2)Jensen et al. Phys. Rev. Lett., 90, 086101 (2003) also observed similar depletion of water near hydrophobic surface with Xray reflectiveity studies. They also correlated their observations with contact angle measurements 3) Langmuir, 23, 598 (2007) used neutron diffraction to observe similar depletion between water and hydrophobic solid surface.}

\label{ID_19652608}\subsubsection{Solvation and capillary waves}

\label{ID_106467715}\paragraph{Consider the change in the free energy of a system when a l-v interface of a solvent is brought in contact with a solute that is fixed in space. Let the entire system be in gran canonical ensemble. Therefore difference in free energy is given as follows:}

\label{ID_1903608681}\paragraph{Here the superscripts f and i denote the state with solute in contact with l-v interface and the one where it is in contact with saturated vapor, respectively. Let us denote all the similar differences in extensive quantities by operator Delta,. Following from the second law, the above equation can then be written in a differential form as follows:}

\label{ID_1377597507}\paragraph{We assume here that the solvent is at liquid-vapor coexistence conditions and that the system is at constant volume. Then the temperature derivative of free energy is given by}

\label{ID_863437212}\paragraph{Note that if the l-v interface was initially not influenced by any external field, then under saturation condition, the free energy change due to its rise upto the solute will be zero. In other words \Delta \mu above, only represents the change due to solute. Therefore \Delta S and \Delta N_s_o_l are mainly due to the fixed solute. The entropy change may be further divided into the contributions coming from the solute itself (i), solute-solvent interactions (uv) and solvent-solvent interactions (v v). The first part depends on the temperature and the intramolecular interactions of the solute. The second part consists of fluctuations in the solute-solvent interaction energy. The final part is due to the changes in the solvent-solvent interactions that are brought about by the solute. Note that the \Delta S_{uv} also contains the indirect effect of solvent-solvent changes on solute-solvent fluctuations. Typically, the "structural" changes in the solvent brought about by the solute are included in the third term, whereas the "effect of fluctuations in the solvent" is accounted in the second term. The density-density correlations along liquid-vapor interface are "fluctuations" and therefore they contrubte to the second term.}

\label{ID_1074911865}\paragraph{Since the free energy change associated with the rise of liquid-vapor interface is zero, we can look at the above process in terms of inserting a solute at the position r in the existing liquid-vapor interface. The related free energy change and its temperature derivative is given as follows:}

\label{ID_388366021}\paragraph{Here v_{\Psi} denotes the partial derivative of the solute-solvent interaction potential with respect to the coupling parameter \lambda. Note that \rho_v (r_1 , r) denotes the density of the vapor around the solute molecule at the initial state corresponding to \Omega_i (r). Starting from the above expressions, \delta S can be expressed as follows: Here, &lt;...&gt; denote the sensemble average taken over the entire system and &lt;..&gt;_1 denote the average taken over a system with solvent molecule fixed at r_1.  Note that \Phi and V are the contributions from the solvent-solvent interactions and solute-solvent interactions, respectively. M denotes the contributions from the chemical potential of the solvent . [TO CHECK: From our previous calculations with water and a model solvent, we observed that the for different distances from the solute \Phi and V are negligible within simulation uncertainties or very small.]}

\label{ID_349834534}\paragraph{The superscripte v denote the same quantities for the initial state. We are assuming here that the solute is fixed and its internal degrees of freedom are not influenced by temperature. Also, we can express M in terms of density-density correlations. The different entropy components can then be expressed as follows: Where, \chi (r_1) is the local compressibility of the solvent at position r_1. Again, the superscripts  refer to the quantity corresponding to the initial state.}

\label{ID_1103070320}\paragraph{In order to understand the second term, We note that \Delta N_{sol} is also related to the compressibility of the solvent at the liquid-vapor interface as follows (refer cap_wave_fields2):  This can be easily derived from the above equations. Since the function v_{\Psi} strongly decreases with the distance from the solute, and the solute is positioned at the liquid-vapor interface, all the integrals above are mostly influenced by solvent moleculecules near the solute.}

\label{ID_1305040262}\paragraph{ It is known that the major contributon to \chi comes from transverse correlations along liquid-vapor interface, so called capillary waves. Therefore CWs affect the temperature dependence predominantly via the terms containing \chi as follows: Further observe that the effect is related to how liquid-vapor saturation properties of the solvent change with temperature. [TO CHECK: For studies with water and a model solvent, we did observe that unlike other terms, this term is statistically significant. Thus, transverse correlations play an important role in the temperature dependence of the free energy change associated with a single solute fixed at the liquid-vapor interface.]}

\label{ID_276381896}\paragraph{ We now turn towards solid surfaces. In the above equations, the solute acts as a static field and therefore, they are even valid for more than one solute molecule fixed at different locations in space. As we gradually increase the density of the solute molecules, the "liquid-vapor interface" will start  resembling the solid surface more. The free energgy change \Delta \mu can then be expressed as follows: Here A is the interfacial area and \gamma_{ij} represents the interfacial free energies between phases i and j. Therefore, to understand the temperature dependence of \gamma_{sv} - \gamma_{sl} for anatomistically detailed solid surface, we should analyze the terms in the above expressions}

\label{ID_374576713}\subsection{Research status (national)}

\label{ID_1169605654}\subsubsection{Jayant Singh's work}

\label{ID_1127102184}\paragraph{Nucleation of ice on nano-patterned surfaces and its relation to wetting}

\label{ID_21321814}\paragraph{Temperature dependence of wetting of water on PNiPAM brushes: Refer "Molecular dynamics study of wetting behavior of grafted thermo-responsive PNIPAAm brushes"}

\label{ID_154250283}\subsubsection{Sanjay Puri}

\label{ID_920584188}\paragraph{Theoretical studies for Phase equilibria near surfaces: J. Chem. Phys. 139, 174705 (2013)}

\label{ID_595749615}\paragraph{Studies on the effect of temperature gradients on the phase behavior of fluids: Europhysics Letters, 103 (2013) 66003}

\label{ID_135785427}\subsubsection{Chinmay's group compares sessile drop and calorimetric approaches}

\label{ID_1532965186}\paragraph{Useful for justifyng the microcalorimetry}

\label{ID_838488932}\paragraph{Can coating density affect the temperature dependence?}

\label{ID_1401479645}\paragraph{Can mix of different coatings be tested to "tune" hydrophilic(phobic) nature?}

\label{ID_1523203023}\subsubsection{Work of Harish C Barsilia concerning the manufacture of super-hydrophobic surfaces: 1) Vacuum (2014), 99, 42-48 2) Solar Energy Materials &amp; Solar Cells (2012), 107, 219-224 Also studies the temperature}

\label{ID_1979164963}\subsection{Novelty of our work}

\label{ID_1997632539}\subsubsection{How realistic our models should be?}

\label{ID_625214648}\subsubsection{Exploiting transverse correlations}

\label{ID_1607535616}\subsubsection{New ways of studying wetting transitions for technologically relevant systems via experiments}

\label{ID_825518109}\paragraph{Emphasis on trends with a particular order parameter rather than measurements}

\label{ID_1028951925}\paragraph{Is it possible to create "a set of simple models" to consider a porous structure?}

\begin{itemize}
\label{ID_1602351942}\item New possible project
\label{ID_714004449}\item Alternatives to characterizing surface with Fowkes equation
\end{itemize}
\label{ID_1602351942}\label{ID_714004449}\label{ID_30899692}\subsubsection{Considering theories developed for flat surfaces in the context of more realistic and applied topic of porous powders}

\label{ID_1615838882}\section{Work plan}

\label{ID_392987275}\subsection{Collection of data for (\gamma_{sv} - \gamma_{sl}) and \frac{d(\gamma_{sv} - \gamma_{sl})}{dT}. The objective here is to determine the trends in the above properties with respect to the parameters related to the solid surface. Experiments will be used to understand the effect of features that can be manipulated on a larger scale (greater than 100nm). The properties at the scale between few nanometers to hundreds of nanometers will be computed from well tested theoretical models. Molecular simulations will be used for features present from the atomistic scale to few nanometers where deviations from the common theoretical models are observed. In the molecular simulation studies, the focus will be more on the variation of the abovementioned properties with features of the solid surface instead of quantitative comparison with real systems. Therefore, the choice of models used for simulations and the employed system conditions will depend on the computational feasibility.}

\label{ID_116998146}\subsubsection{We describe here the information that we plan to obtain from these studies. Let us denote the three surface features by capital letters A, B and C. A "feature" may be related to a particular characteristic of the surface like overall solid-fluid interaction or roughness. A particular feature may be manipulated by changing certain characteristics of the surface. For example, roughness may be manipulated by considering solid surfaces with different roughness factors. Let us say that the number of systems that are used to study the trends with respect to features A, B and C be denoted by a, b and c, respectively. Our first aim is then to collect the data to be represented on the plot of \frac{d(\gamma_{sv} - \gamma_{sl})}{dT} vs \gamma_{sv} - \gamma_{sl} as shown in the adjacent [figure]. In this figure, the points A_{1} to A_{a} represent the data obtained for 'a' systems for determining the effect of surface feature A and, so on. These 'a' points will contain data from simulations as well as experiments depending on the scale at which the feature A is investigated. For example, if A is surface roughness then molecular simulations will be used to study the effect on roughness at scales smaller than few nanometers whereas experiments will be used for larger scales. Also note that the data may contain the results obtained at different temperatures and for different fluids. The data as represented in the [figure] can then help us to characterize the trends observed in  ( \gamma_{sv} - \gamma_{sl},frac{d(\gamma_{sv} - \gamma_{sl})}{dT}) by manipulating each feature. Each of the "features" that we mentioned above are broad categories that can be manipulated by changing different surface characteristics. For example, a rough surface contains  different surface features like hills and valleys. The roughness feature can then be manipulated by changing the depth of the valleys or by changing the spacing between hills. If the points obtained by changing the former characteristic lie on a particular curve on the adjacent diagram then those obtained by changing later characteristic may follow a different curve. This is represented in the adjacent plot where the curves l_{1}^A and l_{2}^B represent the trends obtained by manipulating feature A in two different ways. From the collected data we want to identify such profiles that will help us identify the characteristics of the surface that needs to be modified to vary  \gamma_{sv} - \gamma_{sl} and frac{d(\gamma_{sv} - \gamma_{sl})}{dT} in a particular manner. Since the data also contains the information obtained from different fluids and at diiferent temperatures, the analysis will also provide insights into the variation of the above properties with fluid nature and temperature. Though the main goal of this project is to understand the effect of solid surface, we plan to obtain results with few fluids to undderstand the variation with fluid nature. Typically, these fluids will differ in the proportion of electrostatic and dispersion interactions between theire molecules or ions. For example, experiments will be performed using fluids like hexadecane (lacking electrostatic interactions), water (both electrostatic and nonelectrostatic) and room temperature ionic liquids (electrostatic interactions are dominating). Coming to the molecular simulations, we will use fluid models with different proportion of electrostatic and dispersion interactions. These models will mostly contain monoatomic particles interacting with electrostatic and non-electrostatic interactions, the parameters of which will vary for different fluids. The choice of such a simple model of fluid has two advantages: 1) It ensures that only the "type" of fluid-fluid interaction is considered and excludes the effects due to changes in structure of molecules that are always present when real fluids like water or hexadecane are used. 2) The properties can be computed with lower computational expenses, which is important because we are interested in studying wide variety of systems in this project.  In the following paragraphs we describe the main surface features that we will look at.}

\label{ID_1369049574}\subsubsection{Attraction between the solid surface and fluid. Here the effect of attraction between the molecules of fluid and those of the solid surface will be studied for homogeneous surfaces. The measurements or computations will be performed using single crystal surfaces with different orientations. The interaction between the fluid and surface-atoms will be modified using suitable approaches.  The solid surfaces will be single crystal surfaces of compounds like TiO2 and ZnO. The chemical nature of these surfaces can be tuned by using a suitable radiation. Moreover, the effect of such modification on the  wetting behavior has been demonstrated. For example, exposure to UV radiation makes the solid surface more hydrophilic. Regarding molecular simulations, the solid surfaces will be crystalline and made of monatomic particles. With the techniques to be described later, we can manipulate the interaction between atoms of solid surface and those of fluid molecules to cover all the wetting regimes, from complete drying to complete wetting.}

\label{ID_708499358}\subsubsection{Roughness of the solid surface. The solid surface can be rough at multiple scales. At the lowest scale, the arrangement of molecules on the face of the crystal can be said to characterize a particular form of roughness. In order to study the effect of such roughness, the studies performed in the earlier paragraph will employ solid crystals with different orientation of surface atoms ( say 100, 110 or 001 crystal faces). For larger scale roughness, the experimental system will depend on the scale that is targeted. Note that all the solid surfaces below are "rigid", that is the relative positions of surface molecules is negligibly affected by the liquid or the investigated range of temperatures. We plan to account for the flexibility of the surface and consider surfaces made of thermoresponsive polymers in the future work. In later section we describe the specific tests that will be used to ensure the rigidity of these surfaces. [Advances in Colloids and Interface Science, 227, 57 (2016)]. The solid surfaces will be made of TiO2, ZnO and SiO2 and patterned using plasma [Vacuum (2014), 99, 42-48 2)] or chemical itching [PNAS, 105, 18200 (2008) ]. The duration of the itching process and the conditions employed will enable us to control the roughness. Most of these patterns will lack regular geometric features as shown in [figure]. However, using silica substrates we plan to study few surfaces with regular microscopic patterns as shown in the adjacent [figure]. The use of materials like TiO2 and ZnO will also enable us to vary the interaction between the surface atoms and those of the liquid (see previous section). The scale of features that we expect to study in experimental systems will range from 100nm to few micrometers. Coming to lower scales, It has been shown that the wetting behavior observed for features less than few tens of nanometers is significantly different [Langmuir, 29, 11815 (2013)]. For these scales we will use molecular simulations. The solid used in molecular simulations will be crystalline (FCC, BCC etc.) and made of monoatomic molecules. If the solid surface lies in x-y plane, then the surface patterns can be broadly chracterized into two types: 1) Those that vary along only 1 axis (stripped pattern) and 2) Those that vary along both x and y axes (chessboard pattern).  Note that in molecular simulations, the features are always periodic at macroscopic scale. In each of these categories we will consider features with two different shapes: 1) Sinusoidal and 2) rectangular. The studies will be performed for different characteristic dimensions of each of the above four types of patterns. For example, for the rectangular features, we will repeat the studies for different spacings between the pillars and different heights of the pillars.  Additionally, as in the previous section, we will also see the effect of varying solid-fluid interaction strengths using advanced simulation techniques.}

\label{ID_555034144}\subsubsection{Chemically heterogeous solid surfaces. We first discuss the surfaces for experimental studies. At the atomistic level, chemical heterogeneity is synonimous with the structure of individual molecules forming the surface and their arrangement in crystal. Hence, the measurements to be performed for different crystal faces of TiO2 (see earlier section) will also include the effect of chemical heterogeneity. It is also possible to produce chemical heterogeneity at similar or slightly larger scales by using surfaces made of block co-polymers or by chemical functionalization. However, the rigidity of such surfaces remains to be tested and we expect that their properties will be significantly affected by temperature. Therefore, we will not investigate such heterogeneity in the current project. For scales ranging from 100nm to few mictrometers, we will use TiO2 and ZnO surfaces that are made heterogeneous by selective exosure to radiation. Studies will be performed with two types of patterns : 1) Stripped and 2) Chess board, as shown in the adjacent figure. For each type, we will also consider different dimensions of the features. Turning to molecular simulations, we will use them to cover the features ranging from atomic scale to few nanometers. The solid surface will be crystalline and made of monoatomic molecules. In order to study the atomic scale heterogeneity, we will use different interaction parameters for the atoms  in the liquid-facing part of a unit cell [figure]. On larger scales, we will again consider stripped and chessboard patterns with different dimensions of the features. In this project, the patterns will contain regions of  two types that are distinguished by the interaction parameters between solid and fluid. The patternns containg refions of more than two ctypes will be considered in future studies following the development of computationally effecient methods to handle such scenarios.}

\label{ID_581300306}\subsection{Time schedule}

\label{ID_1195711759}\subsection{Use of research outcome}

\label{ID_1268337594}\subsection{Risk analysis}

\label{ID_3125732}\section{Expertise and preliminary results}

\label{ID_1455861183}\subsection{Computational}

\label{ID_485443609}\subsubsection{For a ghost solute, how ensemble averages of Equation (25) of cap_wave_field1 paper change with temperature for bulk liquid and interface}

\label{ID_577294012}\subsubsection{How the contribution from capillary waves (deldelmu) affected by temperature?}

\label{ID_1474088810}\subsubsection{Studies with 2 solutes with different separation at bulk and interface: 1) How coeffs of Equation (25) ar affected by distance and 2) How they are affected by temperature}

\label{ID_1367637509}\subsubsection{Temperature dependence of wetting behavior on a lattice surface different than the one earlier observed}

\label{ID_1178677590}\subsection{Experimental}

\label{ID_1995511474}\subsubsection{measuring contact angles and temperature dependence on flat surfaces}

\label{ID_853592622}\subsubsection{Measurin wetting properties in porous beds}

\label{ID_1635488399}\subsubsection{Analysis of crystal faces and working with crystals using processes like milling}

\label{ID_1082352398}\section{Money and technical resources}


\newpage
%\tableofcontents

\end{document}
